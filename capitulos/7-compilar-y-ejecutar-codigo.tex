%Para darle formato al código en Haskell (dado que hacerlo de forma manual) es una tanto tedioso se utilizó
%\href{https://github.com/jaspervdj/stylish-haskell}{\texttt{stylish-haskell}}.

%A la hora de buscar ``símbolos'' en de manera más inteligente en \LaTeX se utilizó
%\href{https://github.com/jaspervdj/stylish-haskell}{Detexify}, haciendo un ezbozo del símbolo
%a buscar y nos dará sugerencias de cuál podría ser.


%De igual manera el código en C++ se le dio estiló uttilizando el estándar de \href{https://llvm.org/docs/CodingStandards.html}{LLVM}

%Lazy I/O

%# Explicar la entrada en haskell

%Ver si cambio `fmap` con `<$>`

%- `lines`,
%- Para leer cada test case use una función auxiliar llamada `join`
