En este capítulo se verá como, dado un conjunto de definiciones de funciones, por medio de razonamiento ecuacional podemos llegar a otras definiciones y/o probarlas. Las pruebas aquí se harán mediantte inducción.

Muchas veces es algo engorroso probar funciones similares repetidamente, por eso veremos una forma de hacer pruebas (en algunos casos) más cortas,
presentando unas \textit{funciones de orden superior} que encapsulan patrones comunes de cómputo. Y así, probar resultados más generales y apelar a ellos.

Al final se verá que la eficiencia también importa, porque se mostrarán algunos ejemplos; como un problema famoso llamado \textit{``The maximum segment sum''}
y una mejora de la función \texttt{scanr}. Y todo esto se logrará como consecuencia de lo dicho anteriormente.

\subsection{Inducción sobre listas}
Recordemos que toda lista finita es de la forma; una lista vacía \texttt{[]} ó \texttt{x:xs} donde \texttt{xs} es una lista finita. Por consiguiente, para probar que $P(xs)$ se mantiene para todas las lista finitas $xs$,
se tiene que probar que:

\begin{enumerate}
    \item $P([])$ se cumple
    \item Para toda $x$ y para todas las listas finitas $xs$, que $P(x:xs)$ se cumple dado que $P(xs)$ también.
\end{enumerate}

Tomemos la definición de concatenación \texttt{(++)},
\inputminted{haskell}{definiciones/concatenation.hs}

Y ahora probemos que (++) es asociativa para todas las listas finitas $xs$, es decir:

(xs ++ ys) ++ zs = xs ++ (ys ++ zs)

Por inducción sobre $xs$:

%{\displaystyle x+5} is the left-hand side (LHS) and {\displaystyle y+8}{\displaystyle y+8} is the right-hand side (RHS).
%FIXME: si caben dos eucaciones por lado lo hago en columnas.

\begin{itemize}
\item Caso []
\begin{minted}{haskell}
(LHS)

([] ++ ys) ++ zs
=   {++.1}
ys ++ zs
\end{minted}

\begin{minted}{haskell}
(RHS)

[] ++ (ys ++ zs)
=   {++.1}
ys ++ zs
\end{minted}

\item Caso (x:xs)
\begin{minted}{haskell}
(LHS)

((x:xs) ++ ys) ++ zs
=   {++.2}
(x:(xs ++ ys)) ++ zs
=   {++.2}
x:((xs ++ ys) ++ zs)
\end{minted}

\begin{minted}{haskell}
(RHS)

(x:xs) ++ (ys ++ zs)
=   {++.2}
x:(xs ++ (ys ++ zs))
=   {induction}
x:((xs ++ ys) ++ zs)
\end{minted}

\end{itemize}

\subsection*{La función \texttt{foldr}}
\subsection*{La función \texttt{scanl}}

\begin{minted}{haskell}
scanl (@) e [x, y, z, ...] = [e, e@x,(e@x)@y,((e@x)@y)@z,...]

\end{minted}



\section{Programación funcional}
De forma muy general y resumiendo, la programación funcional:
\begin{itemize}
    \item es un método de construcción de un programa que hace énfasis en las funciones y sus aplicaciones
    en vez de cómandos y sus ejecuciones.

    \item usa notación matemática simple que permite que los problemas sean descritos de manera clara
    y consisa.
    \item tiene bases matemáticas que fundamentan el razonamiento ecuacional acerca de las propiedades de
    los programas.
\end{itemize}

\section{Definiciones inductivas y recursivas}
%TODO

\section{Razonamiento ecuacional}
%TODO


\section{Definiciones de listas}
%TODO

\section{Principio de Fusión}
\begin{minted}{haskell}
f . foldr g a = foldr h b
\end{minted}

\begin{itemize}
    \item $f$ es una función estricta.
    \item $f a = b$
    \item $f(g y x) = h (f y) x$ para toda $x$ y $y$.w
\end{itemize}

\section{\textit{The maximum segment sum}}
