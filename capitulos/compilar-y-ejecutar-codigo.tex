%Para darle formato al código en Haskell (dado que hacerlo de forma manual) es una tanto tedioso se utilizó
%\href{https://github.com/jaspervdj/stylish-haskell}{\texttt{stylish-haskell}}.

%A la hora de buscar ``símbolos'' en de manera más inteligente en \LaTeX se utilizó
%\href{https://github.com/jaspervdj/stylish-haskell}{Detexify}, haciendo un ezbozo del símbolo
%a buscar y nos dará sugerencias de cuál podría ser.


%De igual manera el código en C++ se le dio estiló uttilizando el estándar de \href{https://llvm.org/docs/CodingStandards.html}{LLVM}

%Lazy I/O

%# Explicar la entrada en haskell

%Ver si cambio `fmap` con `<$>`

%- `lines`,
%- Para leer cada test case use una función auxiliar llamada `join`

\section{Compilar y ejecutar código}


As the Camel Book says:
\epigraph{I recall seeing a package to make quotes}{Snowball}

Unless you're using artificial intelligence to model a solipsistic philosopher, your program needs some way to communicate with the outside world.



% TODO: Poner las funciones que uttilicé para procesar el input
% lines y esas https://hackage.haskell.org/package/base-4.14.0.0/docs/Data-List.html 


% TODO: explicar I/O haskell
% https://byorgey.wordpress.com/2019/05/22/competitive-programming-in-haskell-scanner/
% https://byorgey.wordpress.com/2019/04/24/competitive-programming-in-haskell-basic-setup/

% https://wiki.haskell.org/Tutorials/Programming_Haskell/String_IO
% 