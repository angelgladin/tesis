QuickCheck es una biblioteca de Haskell para hacer pruebas aleatorias que deben cumplir ciertas
propiedades nuestros programas. El programador provee una especificación de su programa en la
forma de propiedades las cuales las funciones deben satisfacer y, QuickCheck prueba que esas
propiedades se mantengan en un gran número de casos de casos generados aleatoriamente. Las
especificaciones son expresadas en Haskell usando combinadores proporcionados por QuickCheck.
QuickCheck provee combinadores para definir propiedades, observar la disctribución de los casos
de prueba y, definir los generadores de información generadores.\footnote{
    Descripción de QuickCheck tomada de la documentación oficial
    \url{https://hackage.haskell.org/package/QuickCheck}.
}

\section{Motivación} % TODO: ver si lo dejo


\inputminted{haskell}{codigo/haskell/test-failure-function.hs}
\inputminted{haskell}{codigo/haskell/test-matching.hs}


\section{Experimentos} % TODO: cambiarlo
