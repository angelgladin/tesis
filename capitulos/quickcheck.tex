\epigraph{\itshape Randomized testing exemplifies the 80/20 rule: it yields 80\% of the benefit
of formal verification for 20\% of the effort.}{Matt Might}

{\QuickCheck} es una biblioteca de Haskell para hacer pruebas generadas 
aleatoriamente sobre ciertas propiedades que deben cumplir los programas. El 
programador provee una especificación de su programa mediante propiedades 
que satisfacen las funciones  y, {\QuickCheck} prueba que esas propiedades se 
mantengan en un gran número de casos generados aleatoriamente.\\
Las especificaciones son expresadas en sintaxis de Haskell usando combinadores 
proporcionados por {\QuickCheck} 
% {\QuickCheck} provee combinadores 
para definir propiedades, observar la disctribución de los casos
de prueba y, definir los generadores de información.\footnote{
    Descripción de {\QuickCheck} tomada de la documentación oficial\\
    \url{https://hackage.haskell.org/package/QuickCheck}.
}

\noindent\rule{\textwidth}{1pt}
\hfill \break

En esta sección se verá la importancia de usar {\QuickCheck}; tanto como una 
manera rápida de verificar ciertas propiedades que deben cumplirse en alguna 
implementación que se haya hecho en Haskell, ventajas y desventajas sobre su 
uso y, el porqué es útil utilizarlo y qué problemática aborda.

Se verá en un ejemplo práctico, en este caso tanto en
\hyperlink{funcional:funcion_error}{la función de error} y en
\hyperlink{funcional:kmp}{\textit{Knuth-Morris-Pratt}}, implementados en 
Haskell. En ambos ejemplos se expondrá la versión \textit{naïve} y la versión 
``optimizada'' que fue derivada mediante razonamiento ecuacional. 
Y por medio de casos generados aleatoriamente se analizará el resultado que
{\QuickCheck} proporciona.


\section{Motivación}
Hoy en día el estilo de hacer pruebas a nuestro código es mediante 
\textit{pruebas unitarias (unit testing)} en las cuales; se supone un ``estado 
del mundo'' o del sistema, se ejecutan la pruebas unitarias, se revisa checa si 
el estado modificado del ``mundo'' hace lo que debería hacer y al final se 
confirma que todo el sistema siga funcionando como se debe o como se espera. 

A continuación se muestra un conjunto de aserciones (\texttt{assert}) o 
afirmaciones en Java para ejemplificar esta idea. Una aserción puede ser 
utilizada para verificar si una suposición hecha durante la implementación del 
programa se mantiene válida cuando el programa es ejecutado.

\begin{minted}[frame=lines, framesep=10pt, linenos]{java}
class TestAdder {
    void testSum() {
        Adder adder = new AdderImpl();
        assert(adder.add(1, 1) == 2);
        assert(adder.add(1, 2) == 3);
        assert(adder.add(2, 2) == 4);
        assert(adder.add(0, 0) == 0);
        assert(adder.add(-1, -2) == -3);
        assert(adder.add(-1, 1) == 0);
        assert(adder.add(1234, 988) == 2222);
    }
}

interface Adder {
    int add(int a, int b);
}

class AdderImpl implements Adder {
    public int add(int a, int b) {
        return a + b;
    }
}
\end{minted}

En este tipo de pruebas se pueden agregar gran cantidad de objetos que simulan 
comporamientos (\textit{mock objects}) o, un sinnúmero de casos prueba pero 
¿cuál es el problema aquí? El código anterior solo tiene 7 pruebas y no es 
dificil de imaginar que se pueden crear muchos más casos, o simplemente si se 
implementa otra funcionalidad el espacio de búsqueda de posibles \textit{bugs}
crece de forma exponencial.

Claramente hacer pruebas es impráctico, pero ¿por qué usar {\QuickCheck}?
\begin{itemize}
\item Se pueden combinar las pruebas basadas en propiedades (en la siguiente 
sección se hablará de esto) con casos de prueba generados 
\textit{aleatoriamente}.
\item El programador escribe propiedades a cumplir en vez de pruebas en 
específico.
\item Obviamente no nos puede dar la misma seguridad de hacer pruebas 
exhaustivamente, o mejor aún usar un asistente de pruebas, pero \textbf{es muy 
práctico}.
\item Se puede escoger la cantidad de información que será sometida a nuestras 
propiedades.
\end{itemize}


\section{Uso básico}

Consideremos la siguiente propiedad sobre \textit{listas finitas} que usa la 
función \hsCode{reverse}:
\begin{minted}{haskell}
prop_rev_rev :: [Int] -> Bool
prop_rev_rev xs = reverse (reverse xs) == xs
\end{minted}

Después de haber importado el módulo \hsCode{Test.QuickCheck} en \texttt{ghci}, 
se prueba con 100 listas aleatorias usando 
\hsCode{quickCheck :: Testable prop => prop -> IO ()}.
\begin{minted}{text}
>>> quickCheck prop_rev_rev
+++ OK, passed 100 tests.
\end{minted}

Y aunque esta propiedad es verdadera (que se demostrará enseguida) se puede 
ver lo útil y práctico que es {\QuickCheck} en este ejemplo.

\begin{proof}
Por inducción estructural sobre listas. Demostrar que,\\
\hsCode{reverse (reverse xs) = xs} para toda lista finita \texttt{xs}.
\hfill \break
Recordando la función \hsCode{reverse} definida como:

\begin{minted}{haskell}
reverse :: [a] -> [a] -> [a]
reverse [] = []                     -- (reverse.1)
reverse (x:xs) = reverse xs ++ [x]  -- (reverse.2)
\end{minted}

\begin{itemize}
\item Caso base:
\begin{minted}{haskell}
reverse (reverse [])
  = -- {reverse.1}
reverse []
  = -- (reverse.1)
[]
\end{minted}

\item Caso inductivo, demostrar para \hsCode{reverse (reverse (x:xs)) = x:xs}.
\begin{minted}{haskell}
reverse (reverse (x:xs))
  = -- {reverse.2}
reverse (reverse xs ++ [x]))
  = -- {Distributividad de reverse}
reverse [x] ++ (reverse (reverse xs))
  = -- {Propiedad de reverse en lista unitaria}
[x] ++ reverse (reverse xs)
  = -- {Hipótesis de inducción}
[x] ++ xs
  = {Por (++)}
x:xs
\end{minted}
\end{itemize}


\begin{proof}
Por demostrar que la distributividad de \hsCode{reverse} sobre la 
concatenaci´on de listas es válida, pero que el orden de los argumento es 
intercambiado:
\hsCode{reverse (xs ++ ys) = reverse ys ++ reverse xs} para toda lista finita 
\texttt{xs}.

Como el operador de concatenación \hsCode{++} está definido con caza de patrones 
sobre el primer argumento, es natural verificar esta propiedad con inducción 
sobre \texttt{xs}.
Recordando la función \hsCode{(++)} definida como:

\begin{minted}{haskell}
(++) :: [a] -> [a] -> [a]
[] ++ ys = ys                  -- (++.1)
(x:xs) ++ ys = x : (xs ++ ys)  -- (++.2)
\end{minted}

\begin{itemize}
\item Caso base:
\begin{minted}{haskell}
reverse ([] ++ ys)
  = -- {(++.2)}
reverse ys
  = -- {Identidad derecha de concatenación}
reverse ys ++ []
  = -- {Por la igualdad de reverse.1}
reverse ys ++ reverse []
\end{minted}

\item Caso inductivo:
\begin{minted}{haskell}
reverse ((x:xs) ++ ys)
  = -- {Por ++.2}
reverse (x : (xs ++ ys))
  = -- {Por reverse.2}
reverse (xs ++ ys) ++ [x]
  = -- {Hipótesis de inducción}
(reverse ys ++ reverse xs) ++ [x]
  = -- {Asociatividad de ++}
reverse ys ++ (reverse xs ++ [x])
  = -- {Por la igualdad de reverse.2}
reverse ys ++ reverse (x:xs)
\end{minted}
\end{itemize}

\end{proof}

Concluyendo así que la función \hsCode{reverse} es su propia inversa.
\end{proof}

Ahora consideremos un caso donde la propiedad falle en {\QuickCheck}, y se 
mostrará un contraejemplo.
\begin{minted}{haskell}
prop_wrong :: [Int] -> Bool
prop_wrong xs = reverse xs == xs 
\end{minted}

Mostrando así un caso donde no cumple la propiedad.
\begin{minted}{text}
>>> quickCheck prop_rev_id
*** Failed! Falsified (after 5 tests and 3 shrinks):
[1,0]
\end{minted}

\noindent\rule{\textwidth}{1pt}

Veamos otro ejemplo, para todo entero $n$, si $n$ es par entonces $n+1$ es 
impar. Aunque la demostración es trivial, con {\QuickCheck} se puede validar 
(mas no verficar formalmente) con varios números generados aleatoriamente para 
probar esta propiedad.
\begin{minted}{haskell}
>>> quickCheck (\n -> even n ==> odd(n + 1))
+++ OK, passed 100 tests
\end{minted}
donde \hsCode{==>} es una propiedad condicional que descarta los casos que no 
satisfacen esa
precondición.

\noindent\rule{\textwidth}{1pt}

Cuando se escriben propiedades se debe tener en cuenta la siguiente información 
(donde la última es la más importante):
\begin{itemize}
\item Las propiedades son simples funciones booleanas.
\item La convención es que los nombres de las propiedades tienen el prefijo 
\hsCode{prop_}.
\item Los parámetros están ímplicitamente cuantificados con un cuantificador 
universal ($\forall$).
\item Las funciones polimórficas deben de estar restringidas a un tipo
en específico.
\end{itemize}


\section{Generadores}

Aunque {\QuickCheck} puede generar algunos valores aleatorios para algún tipo 
ímplicitamente, hay ocasiones que es m[as productivo definir valores aleatorios  
propios. 
%FIXME falta explicar quien es Arbitrary...
Es aquí cuando se debe escribir una instancia de \hsCode{Arbitrary} 
para esto. Los valores que se ocupan en la pruebas son producidos por los 
\textit{generadores de valores}.

Los Generadores tienen tipos de la forma \hsCode{Gen a}, que es un generador de 
valores para un tipo \hsCode{a}. El tipo \hsCode{Gen} es una mónada, así que se 
puede utilizar la notación \hsCode{do} y las funciones monádicas pueden ser 
utilizadas para definir generadores.

Veamos el siguiente ejemplo; consideremos el tipo de dato algebraico 
\hsCode{Point} que representa un punto en un espacio bidimensional. Lo que se 
quiere hacer es generar puntos con coordenadas $x$ y $y$ aleatoriamente.

\begin{minted}{haskell}
data Point = Point Int Int

instance Arbitrary Point where
  arbitrary = do
    x <- arbitrary
    y <- arbitrary
    return (Point x y)
\end{minted}
donde \hsCode{arbitrary :: Gen a} es un generator para valores de un cierto 
tipo, y que en este caso son de tipo \hsCode{Int}.


\section{Probando las implementaciones}

En esta sección revisaremos las versiones \emph{naïve} y las 
obtenidadas por medio de razonamiento ecuacional, tanto en la 
\hyperlink{funcional:funcion_error}{función de error} como en 
\hyperlink{funcional:kmp}{\textit{Knuth-Morris-Pratt}} son probadas uno a uno 
con diferentes casos prueba en Haskell.

Al hacer los casos prueba usando el tipo \hsCode{String} es poco probable 
tener \textit{una buena disctribución} de los caracteres que comprende la 
cadena y, en especial en probar el algoritmo KMP con las cadenas del patrón y 
texto. Si por ejemplo, restringimos las cadenas a que sólo contengan 
caracteres ASCII aún así no serían muy útiles en las pruebas. Consideremos
el siguiente muestreo.

\begin{minted}{haskell}
>>> sample $ vectorOf 10 arbitraryASCIIChar
\end{minted}
La función \hsCode{sample :: Show a => Gen a -> IO ()} genera algunos valores de 
ejemplo y los imprime en la salida estándar. Si se toma algún valor que se 
imprimió, que por ejemplo sería \mintinline{text}|"\t\n]*]i_\fm\SYN"| se ve que 
no es muy útil para nuestro propósito. 
Pero ¿no sería más mejor definir nuestros propios valores aleatorios?

\hfill \break
En los generadores presentados a continuación para la función de error y el 
algoritmo KMP es bastante usual que se trabaje sobre un alfabeto finito 
$\Sigma$; puede incluir de las letras del alfabeto inglés 
$\Sigma = \{ \texttt{a} \ldots \texttt{z} \}$, un alfabeto binario
$\Sigma = \{ 0, 1 \}$ o un alfabeto de ADN donde 
$\Sigma = \{ \texttt{A,C,G,T} \}$ por mencionar algunos. 
De esta manera, se restringen los caracteres tomados con respecto a algún 
alfabeto definida por nosotros.

A continuación se mostrará como restringir las cadenas generadas sobre algún 
alfabeto y su su longitud, con su respectiva implementación.

\newpage
\subsubsection{Función de error}
\inputminted[frame=lines, framesep=10pt, linenos]{haskell}
    {codigo/haskell/test-failure-function.hs}

\begin{itemize}
\item En la línea \texttt{3} y \texttt{4} se importan los módulo de la versión 
de error, tanto la versión que tiene complejidad $O(n^3)$ y $\Theta(n)$ 
respectivamente, que proveerán la función \hsCode{ptable} que es la función de 
error.
\item En la línea \texttt{6} del módulo \hsCode{Data.Array} solo se importa la 
función\\
\hsCode{elems :: Array i e -> [e]} para obtener una lista de elementos del 
arreglo en el orden indexado.
\item En la línea \texttt{8} se hace un \emph{wrapper} para \hsCode{String} con 
el constructor \hsCode{S}. Éste será usado para poder usar la instancia de 
\hsCode{Arbitrary}.
Es mucho mejor definir un nuevo tipo isomorfo a (en este caso) \hsCode{String} y 
así poder definir su generador para este caso en específico.
\item En la línea \texttt{11} a \texttt{15}, para poder usar \hsCode{arbitrary} 
en el tipo \hsCode{Pattern} definido por nosotros, se necesita declarar como 
instancia de \hsCode{Arbitrary}.
En la línea \texttt{12} por medio de la notación \hsCode{do} cada línea será un 
valor monádico, primero se liga el valor \hsCode{size} a un número aleatorio 
entre 1 y 1000 con\\
\hsCode{chooseInt :: (Int, Int) -> Gen Int} que genera un elemento aleatorio 
dado un rango inclusivo, acto seguido se liga la variable \hsCode{text} con una 
cadena del tamaño de la variable \hsCode{size} sobre un alfabeto de la letra 
\hsCode{'a'} a la \hsCode{'z'} con la ayuda de \hsCode{elements :: [a] -> Gen a} 
que genera un elemento de los valores dados (y que la lista debe ser no vacía), 
y así con\\
\hsCode{vectorOf :: Int -> Gen a -> Gen [a]} que genera una lista de una 
longitud dada. Finalmente con la función \hsCode{return} pone el resultado en un 
contexto monádico.
\item En la línea \texttt{17} no es necesario que sea instancia de 
\hsCode{Show}, pero cuando se usa \hsCode{verboseCheck} en \hsCode{QuickCheck} 
es útil para poder mostar los casos prueba.
\item De la línea \texttt{20} a la \texttt{24} primero se extrae la cadena 
\hsCode{as} del constructor \hsCode{S}. 
Después en la expresión \hsCode{let ... in ...} en la línea \texttt{22} se 
transforma lo que devuelve la función \hsCode{ptable} a una lista de tipo
\hsCode{[(Char, Int)]} donde la $i$-ésima posición donde la primera entrada es 
el $i$-ésimo caracter de la cadena y la segunda entrada es la $i$-ésima 
posición de la función de error.
En la línea \texttt{23}, lo que regresa 
\hsCode{FailureFunctionOptimizedp.ptable} regresa
\hsCode{Array Int (a, Int)} pero con la función \hsCode{A.elems} se obtiene un 
lista, que finalmente en la línea \texttt{24} se compara que ambas listas sean 
iguales.
\item Finalmente en la función \hsCode{main} es donde con ayuda de la función 
\hsCode{quickCheck} se hará la prueba de las propiedades y se mostrará el 
resultado en la salida estándar.
\end{itemize}

Para ejecutar las pruebas se pueden hacer directamente en \texttt{ghci}, pero en 
este caso se ejemplificará el comando \texttt{runhaskell} de la siguiente 
manera:

\texttt{\$ runhaskell test-failure-function.hs}

\noindent\rule{\textwidth}{1pt}
Para poder visualizar algunos valores con los que fueron probadas ambas 
funciones, en vez de usar \hsCode{quickCheck} se utilizó \hsCode{verboseCheck}.

\begin{minted}{text}
Running failure function tests...
Passed:
jwvtllqfgknosijmqccjmpwrfjjbaxgqakfhkxvhwjgukoeoipneicodymknicywsrwjwiz
rzoeufakcwyojquaugelxiujdbrexedtmsqhoofquvrucbvvwmjpbeeamiajfrrruvwjixl
pvrrntqyxyopxyoynhpghrknnrikcukgyjamnngkytmqmdmydzqeendgktbltsfdqwhjfhn
cnroyrelnsskwereataldovaxpneqmvnjaxengwvlknppfzrlmvpyeatvhvbatesajcfpcf
upltdezxedaggbhavklpiyjzwk

Passed:
ovfpnmsesopulhcnpjzhsubeuwfsrvcrktozf

Passed:
lgcnmiwzyvtlkpsuukuemwawrdptrulmchwmtqkknymyfnyurmobjhgkpcfrxbxfxhjerks
mehenlcjjwbbkzhxdotgxwasmxgwpmsnhlsoyfshqclxscaxytdwywxozhvqpbkshf

Passed:
wmdtscgkwkkuuxwqmmvdexwvtdplthvgtjbcacxmnncqlnzhfnskrfaynnvcogyqllfkzcc
gwtuyjxlstlrvphafeajgjwwahyqnqgfutvtsfqeqiulljouzkjgvtwgrrnhfpznugyawox
ypqwiubmkylfhrwnbdfkhbnnqpnrmebivpbypkswtnrrmitdgxwqjmfoetqhwuqbbghvmko
wscxbzdl

Passed:
ruhtwpwesozzglmsvzzshonujhutwjkagfhzbdvsdiczpeuzxgwmkugjgpbynlp

...
...
...

+++ OK, passed 100 tests.
\end{minted}

\newpage
\subsubsection{Knuth-Morris-Pratt}
\inputminted[frame=lines, framesep=10pt, linenos]{haskell}
    {codigo/haskell/test-matching.hs}
% TODO: ya solo me falta explicar esto
\begin{itemize}
\item En la línea \texttt{7} se declara un nuevo tipo de dato 
\hsCode{TextPattern} que encapsulará el texto y el patrón a buscar.
\item De la línea \texttt{10} a la \texttt{16} se declara como una instancia de 
\hsCode{Arbitrary}.
La diferencia a notar con el código anterior, es el alfabeto sobre el cual se 
generarán las cadenas que aquí es $\Sigma = \{ \texttt{A,C,G,T} \}$. Y que son 
dos cadenas las generadas.
\item En cada una de las funciones que su nombre tiene de prefijo 
\hsCode{prop_matches} se compara cada una de las implementaciones de buscar un 
patrón en un texto, se compara desde la versión naïve contra versión 
Morris-Pratt y Knuth-Morris-Pratt.
\item Finalmente en la línea \texttt{34} se hace las pruebas de cada una de las 
funciones. 
\end{itemize}

Para ejecutar las pruebas se pueden hacer directamente en \texttt{ghci}, pero en 
este caso se utilizará \texttt{runhaskell} de la siguiente manera:

\texttt{\$ runhaskell test-matching.hs}

\noindent\rule{\textwidth}{1pt}
Para poder visualizar algunos valores con los que fueron probadas ambas 
funciones, en vez de usar \hsCode{quickCheck} se utilizó \hsCode{verboseCheck}.

\begin{minted}{text}
Testing NaiveMatching with MP1...
Passed:
Pattern:GCCGAACCATCGAGTGTATC
Text:TAAGACAATTCAACTCCGAGCTGCGCGGCGCAGTTTGTCACAGTGTGGGGGTGGCAGCTACCCCCTAAG
CGTTATTTATTGGTGTACGCTGAGTATCACAATATTCCACCACCCTCCGTTATGGTTAGTTGATAGAAATTCGA
CCGATGTTTGATTCTTCTAAAGCCAGTGTAACTGCGGATAAGCTGGCTTGG

Passed:
Pattern:CCA
Text:TGCATTAGCGGGTGAGGAGCTCTACGCTTCGCTTTCAGCTTTAAACGATTATGTGCGGTCACCCCAATT
GTCTTATGATGGCCAATATTTCGTCAGTACCACTTCAGTGGCATGACTGAGTGACGCA

Passed:
Pattern:C
Text:CACCACGCCGATTCCTGGTCGTTACCTTAGCCAAATAGACTTAAGTGTTGCTTGACCTGTC

Passed:
Pattern:CGTTGTG
Text:GGGTATCATGACTCCGAAGGATTTGTGACGTTGTGAGGATTTACACCCCGTGAGTCGATCTTTCCTGGT
CTTGATGATGAGATGCGGCACGAAGGTAGGGACTACTTAAGAGGGTCGCTCAAGTAGGATCTGGACGTGTACAG
ACTATCACACCTCTCGCGCCGTACTACTGTTAATCCCCTCTCTCTTGCCATCTGAGCTAGCGCTCAGCTCGATA
TTTGTTTTATA

Passed:
Pattern:AG
Text:AAAAGCGGAACTCAATCTGCCAAGCGCCCGAGCACCTGTGGCCTCGCACTGTGGTGGCGAAGTGTTAAG
GCATGAACCTTATGCCGTAGCAACTTGCACACATACTTTGCATCTAACCTCCACTTAATCCGCGGCGATTGCAT

...
...
...

+++ OK, passed 100 tests.

Testing MP1 with MP2...
Passed:
Pattern:TACGCTTCCGCCTCGTAT
Text:GGTTGCTACACCCCGTCTGCCACCGATACATGGTAGCACTACTTAGCTATGGAGTCCGAAAGAAACCAG
GTATCCGGTAGAATGGATTTAGCGAGGTAATAGGGTCCAGCTTTCAAGCTCAACCCATC

Passed:
Pattern:AGG
Text:CCCGGCTATTGTCCCAAAGCATGATCATAACTTCAAGCTGTGTCCCCTTCTCAGTCGGACAGCGCTAGA
ATCCGCCCGTAGGCAGAACGAAGTGATCGATTTACAGTGTTTTGAAAAGGCCGTCCTGAAACACGCGTCTGTTA
TGTAATTGTGGTATCTTCTACTAGATTTGTGTCATCA

...
...
...

+++ OK, passed 100 tests.

Testing MP2 with KMP...
Passed:
Pattern:CCC
Text:CCTCGACAACAGGTATTCAAGTGTGCATCCATCTTATAGAAGTACATTCAATAACCGTCTAAGCTGCTG

Passed:
Pattern:CGG
Text:AGGCCAGAGCCATTGAACTGAAGAAATAAAGCGTGGCACGACGCACCACCTGCCGCATCTCTCGGCGCA
CACGTTGTGTCTGTGTCCGGTAAAAAATACATATAATCCATACATGTT

...
...
...

+++ OK, passed 100 tests.

Testing NaiveMatching with KMP...
Passed:
Pattern:CCGCATCTGGAGTGG
Text:TCCTGACGGATGAATAGTCTGCCCGACGATTCTTGCTTGACTGCTTATGGTGGTACACTCGTAGAGGGT
GGGGACGTGCTTCAGACCTTAACAGCTTATCGTCTCTGTGTGAACGGTACACGCGACGATACATCGTCCGTAGG
ATTTGCACGGGCGTTCCGCATCTGGAGTGGCTAGATTCAGTTGTAA

Passed:
Pattern:TCA
Text:TCGGAAGGCGACGAGTGCGATAAGAGTCAGCAATAGTTGTGCCTAAGCCATATGATGCCAGACCGCCGT
CATCACTCTCGTAGTCAGGCACTTGAGCTTAAGGTTATCAATTTATCGGGTATTTTTGACATGGCAGGTAAGGG
GGATCGCTTTAGCATGTTAGTAACCCGAAAGACATGATTTTGACATCTCACACGTAGTTATGGAAGCAACTGCC
CTACGGCTCCAGGCGAGCCAACTGCGCCTAGTGGGTGTGCCTGCGCTTAACGACCACTGATTGGTTAGTCTGAT
CCACGAGGGCGGCCAAGGACAAGCATAGACCTGGGCCCCCCTCGGTGCGGCACTAAGATATATCCTGATGGACC
ACCGATATTTTAATGTGGGCACCTACAGGATTCATATGTCAATTATACGATAGATAAAGGATGACCGGGACACA
CATCGA

...
...
...

+++ OK, passed 100 tests.
\end{minted}

A pesar de que se pueden hacer cosas más sofisticadas, y aún mejor hasta llegar 
a utilizar la extensión del lenguaje de Haskell 
(\hsCode{LANGUAGE TemplateHaskell}) para poder probar todas las funciones con el 
prefijo \hsCode{prop_}, por la naturaleza de las pruebas que se requieren no se 
necesita indagar más en las funciones que tiene {\QuickCheck}.

{\QuickCheck} es una biblioteca enorme, que sin duda es bastante útil.
