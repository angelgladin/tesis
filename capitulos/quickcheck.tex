\QuickCheck es una biblioteca de Haskell para hacer pruebas generadas aleatoriamente que deben
cumplir ciertas propiedades nuestros programas. El programador provee una especificación de su
programa en la forma de propiedades las cuales las funciones deben satisfacer y, \QuickCheck prueba
que esas propiedades se mantengan en un gran número de casos de casos generados aleatoriamente. Las
especificaciones son expresadas en Haskell usando combinadores proporcionados por \QuickCheck.
\QuickCheck provee combinadores para definir propiedades, observar la disctribución de los casos
de prueba y, definir los generadores de información generadores.\footnote{
    Descripción de \QuickCheck tomada de la documentación oficial\\
    \url{https://hackage.haskell.org/package/QuickCheck}.
}

\noindent\rule{\textwidth}{1pt}

En esta sección se verá la importancia sobre usar \QuickCheck; tanto como una manera rápida de
verificar ciertas propiedades que deben cumplirse en alguna implementación que se haya hecho en
Haskell, como ventajas y desventajas sobre su uso. El porqué es útil utilizarlo y qué problemática
aborda.

Se verá como será utilizado en un ejemplo práctico; en este caso tanto en
\hyperlink{funcional:funcion_error}{la versión de error} y en
\hyperlink{funcional:kmp}{\textit{Knuth-Morris-Pratt}} en Haskell. En ambos ejemplos se expondrá la
versión \textit{naïve} y la versión ``optimizada'' que fue derivada mediante razonamiento
ecuacional. Y por medio de casos generados aleatoriamente se analizará el resultado que
\QuickCheck proporciona.

\section{Motivación}
Hoy en día el estilo de hacer pruebas a nuestro código es mediante \textit{pruebas unitarias (unit
testing)} en las cuales: se inveta un ``estado del mundo'', se ejecuta la prueba unitaria, se
checa si el estado modificado del ``mundo'' hace lo que debería hacer y al final se ve si todo
sigue funcionando como se debe.

A continuación se muestra un conjunto de aserciones (\texttt{assert}) en Java para ejemplificar
la idea. Una aserción puede ser utilizada para verificar si una suposición hecha durante la
implementación del programa se mántiene válida cuando el programa es ejecutado.

\begin{minted}[frame=lines, framesep=10pt, linenos]{java}
public class TestAdder {
    public void testSum() {
        Adder adder = new AdderImpl();
        assert(adder.add(1, 1) == 2);
        assert(adder.add(1, 2) == 3);
        assert(adder.add(2, 2) == 4);
        assert(adder.add(0, 0) == 0);
        assert(adder.add(-1, -2) == -3);
        assert(adder.add(-1, 1) == 0);
        assert(adder.add(1234, 988) == 2222);
    }
}

interface Adder {
    int add(int a, int b);
}
class AdderImpl implements Adder {
    public int add(int a, int b) {
        return a + b;
    }
}
\end{minted}

En este tipo de pruebas se pueden agregar gran cantidad de objetos (\textit{mock objects}) que
simulan comporamientos o, un sinnúmero de casos prueba pero ¿cuál es el problema aquí? El código
anterior solo tiene 7 pruebas y no es dificil de imaginar que se pueden crear muchos más casos, o
simplemente si se implementa otra funcionalidad el espacio de búsqueda de posibles \textit{bugs}
crece de forma exponencial.

Claramente hacer pruebas es impráctico, pero ¿por qué usar \QuickCheck?
\begin{itemize}
\item Se pueden combinar las pruebas basadas en propiedades (en la siguiente sección se hablará
de esto) con casos de prueba generados \textit{aleatoriamente}.
\item El programador escribe propiedades a cumplir en vez de pruebas en específico.
\item Obviamente no nos puede dar la misma seguridad de hacer pruebas exhaustivamente, pero
es muy práctico.
\item Se puede escoger la cantidad de información que será sometida a nuestras propiedades.
\end{itemize}

%%% HASTA AQUÍ TODO LUCE BIEN!

\newpage
\section{Uso básico}
% TODO: agregar más texto
Consideremos el siguiente ejemplo:
\begin{minted}{haskell}
import Test.QuickCheck

prop_rev_rev :: [Int] -> Bool
prop_rev_rev xs = reverse (reverse xs) == xsq
\end{minted}

% TODO: agregar más texto
\begin{minted}{haskell}
>>> quickCheck prop_rev_rev
+++ OK, passed 100 tests.
\end{minted}

% TODO: agregar más texto
Para todo entero $n$, si $n$ es par entonces $n+1$ es impar. Con \QuickCheck se puede validar (mas
no verficar formalmente) con varios números generados aleatoriamente para probar esta propiedad.
\begin{minted}{haskell}
>>> quickcheck (\n -> even(n) ==> odd(n + 1))
+++ OK, passed 100 tests
\end{minted}

donde \hsCode{==>} descarta los casos que no satisfacen esa precondición

Ahora consideremos un caso donde la propiedad falle. \QuickCheck mostrará un contra ejemplo. Por
ejemplo:
\begin{minted}{haskell}
import Test.QuickCheck

prop_wrong :: [Int] -> Bool
prop_wrong xs = reverse xs == xs 
\end{minted}

% TODO: agregar más texto
\begin{minted}{haskell}
>>> quickCheck prop_rev_id
*** Failed! Falsified (after 5 tests and 3 shrinks):
[1,0]
\end{minted}

Cuando se escriben propiedades;
\begin{itemize}
\item Las propiedades simples son función que son funciones booleanas.
\item La conveción es que los nombres de las propiedades tienen el prefijo \hsCode{prop_}.
\item Los parámetros están ímplicitamente cuantificados con un cuantificador universal $\forall$ y
la última y más importante es que las funciones polimórficas deben de estar restringidas a un tipo
en específico.
\end{itemize}

\section{Generadores}

\section{Probando las implementaciones}

% TODO: ver si quito esto
Es aquí cuando las versiones naïve y las obtenidadas por medio de razonamiento ecuacional,
tanto en la \hyperlink{funcional:funcion_error}{versión de error} como en
\hyperlink{funcional:kmp}{\textit{Knuth-Morris-Pratt}} son probadas uno a uno con diferentes casos
prueba. % TODO: poner algo más aquí.

\subsection{Mónada \texttt{Gen}: un combinador para construir generadores aleatorios}

\subsection{Definiendo un generador}

% TODO: explicar mejor esto.
\hsCode{vectorOf :: Int -> Gen a -> Gen [a]} Genera una lista de una longitud dada.
\hsCode{chooseInt :: (Int, Int) -> Gen Int} Genera un elemento aleatorio dado un rango inclusivo.

\hsCode{elements :: [a] -> Gen a} Genera un elemento de los valores dados. La lista debe ser no vacía.

\newpage
\inputminted[frame=lines, framesep=10pt, linenos]{haskell}
    {codigo/haskell/test-failure-function.hs}
\newpage
\inputminted[frame=lines, framesep=10pt, linenos]{haskell}
    {codigo/haskell/test-matching.hs}

% TODO: explicar aquí porque ttomé lo del ARN

% TODO: completar con información de aquí
% \footnote{TODO \url{https://en.wikipedia.org/wiki/Nucleic_acid_sequence}.}
% http://www.cse.chalmers.se/~rjmh/QuickCheck/manual.html
% https://www.dcc.fc.up.pt/~pbv/aulas/tapf/handouts/quickcheck.html
% http://www.scs.stanford.edu/11au-cs240h/notes/testing-slides.html#(14)
% https://hackage.haskell.org/package/QuickCheck-2.14.2/docs/Test-QuickCheck.html#g:5
% http://matt.might.net/articles/quick-quickcheck/