\label{chap:jueces}

En esta sección se resolverán 2 problemas relacionados con \textit{string matching} usando el
algoritmo KMP y la función de error de la que se habló en el capítulo 4. Se resolverán los
problemas usando C++ y Haskell y se compararán ambas versiones.

%% TODO: Explicar los constraints

\section{SPOJ}
SPOJ (Sphere Online Judge)
% https://wiki.haskell.org/SPOJ
% Tomar algo del competitive aquí

\subsection{Encontrar el factor de repetición de una cadena}
Recordemos en el capítulo 3 \hyperlink{repetition_factor}{el problema 32.1}, es aquí cuando lo
bonito de la programación competitiva y resolver ejercicios se juntan. Ése problema es lo mismo
a resolver lo siguiente y aún mejor, en un juez en línea que puede ``probar'' la implementación.
La especificación del problema dice lo siguiente: 


\includepdf[pages=-]{problemas/pdf/FINDSR.pdf}

\inputminted[linenos, frame=lines]{cpp}{problemas/cpp/FINDSR.cpp}
\pagebreak

En la línea 2 agregamos \texttt{vector} % TODO: explicar aquí qué pedo
De línea 6 a la 19 es la implementación de la función de error.
En la línea 23 es la parte importante sobre leer la entrada; una \textit{heurística} a seguir es que
siempre que diga: ``La entrada contiene varios casos de prueba, cada uno de ellos está descrito en una sola línea'', lo 
que se significa es que se seguirán leyendo valores de \texttt{cin} en la variable previamente declarada \texttt{x} simpre y cuando
el valor siga siendo leído se seguirá en el ciclo \texttt{while}.
% TODO: ecplicar que no nos fijamos en la cota de 10^5 caracteres
También menciona que la última entrada contendrá un solo asterisco \texttt{*} y es ahí cuando se parará de leer de la entrada estandar,
es por eso que primero se lee el valor y despué se compara que no sea diferente.

En la línea 24 se crea un vector cargado con la función de error con el patrón \texttt{s}.

La parte interesante aquí es en la línea 28 porque se obtiene la posible longitud de la $k$-ésima
raíz del patrón. Y de la línea 30 a 33 es cuando se hace la validación previamente analizada.

% TODO: acomodarlos  chido
\begin{table}[]
\begin{tabular}{l|l|l|l|l|l|l|l|l|l|l|l|l|}
\cline{2-13}
$i$        & 1         & 2         & 3         & 4          & 5         & 6         & 7         & 8         & 9         & 10        & 11        & 12        \\ \hline
$P[i]$   & \texttt{a} & \texttt{b} & \texttt{c} & \texttt{a} & \texttt{b} & \texttt{c} & \texttt{a} & \texttt{b} & \texttt{c} & \texttt{a} & \texttt{b} & \texttt{c} \\ \hline
$\pi[i]$ & 0         & 0         & 0         & 1          & 2         & 3         & 4         & 5         & 6         & 7         & 8         & 9         \\ \cline{2-13} 
\end{tabular}
\end{table}

\begin{table}[]
\begin{tabular}{l|l|l|l|l|l|l|l|l|l|l|l|}
\cline{2-12}
$i$        & 1         & 2         & 3         & 4         & 5         & 6         & 7         & 8         & 9         & 10        & 11        \\ \hline
$P[i]$   & \texttt{a} & \texttt{b} & \texttt{c} & \texttt{d} & \texttt{e} & \texttt{f} & \texttt{g} & \texttt{h} & \texttt{0} & \texttt{1} & \texttt{2} \\ \hline
$\pi[i]$ & 0         & 0         & 0         & 0         & 0         & 0         & 0         & 0         & 0         & 0         & 0         \\ \cline{2-12} 
\end{tabular}
\end{table}

\begin{table}[]
\begin{tabular}{l|l|l|l|l|l|l|l|l|l|l|}
\cline{2-11}
$i$        & 1         & 2         & 3         & 4         & 5         & 6         & 7         & 8         & 9         & 10        \\ \hline
$P[i]$   & \texttt{a} & \texttt{a} & \texttt{a} & \texttt{a} & \texttt{a} & \texttt{a} & \texttt{a} & \texttt{a} & \texttt{a} & \texttt{a} \\ \hline
$\pi[i]$ & 0         & 1         & 2         & 3         & 4         & 5         & 6         & 7         & 8         & 9         \\ \cline{2-11} 
\end{tabular}
\end{table}

\begin{figure}[h]
\centering
\includegraphics[width=\textwidth]{spoj/FINDSR-accepted-cpp-haskell}
\caption{El código fue aceptado por el juez en Haskell y C++}
\end{figure}

\inputminted[linenos, frame=lines]{haskell}{problemas/haskell/FINDSR.hs}
\pagebreak


En la línea 1 % TODO: explicar ahí qué pedoo

De la línea 19 a 29 es la implementación de función de error vista en el capítulo % TODO: poner vínculo al pendejo capótulo.

Analisemos parte por partes la línea 4, recordemos la función \\
\hsCode{interact :: (String -> String) -> IO ()} %% Ayudarme con lo de spoj

% TODO: está dlv mi redacción, mejorarla.
De la línea 9 a la 16 es la parte donde se hace el procesamiento del algoritmo principal,
en la expresión \hsCode{let ... in} % TODO: ver si así se podría poner
en la línea 10 se calcula la función de error de la cadena de entrada
en la línea 11 la función \hsCode{bounds :: Array i e -> (i, i)} devuelve los límites del arreglo en cual fue construido y nos quedamos
con la segunda entrada usando \hsCode{snd} para obtener el tamaño del arreglo.
En la línea 12 primero con la función \hsCode{(!) :: Ix i => Array i e -> i -> e} accedemos a la posición \texttt{m} del arreglo y recordando
la función \hsCode{ptable} la segunda entrada es el valor númerico del arreglo.
Acto seguido en la línea 13 oobtenemos  la posible longitud de la $k$-ésima
raíz del patrón. Y de la línea 14 a 15 es cuando se hace la validación previamente analizada.

% TODO: explicar los ejemplos del problema

\subsection{Ver si una cadena es una rotación cíclica de otra}
\hyperlink{cyclic_rotation}{el problema 32.4-7}
\includepdf[pages=-]{problemas/pdf/EC_WORLD.pdf}

\inputminted[linenos, frame=lines, fontsize=\footnotesize]{cpp}{problemas/cpp/EC_WORLD.cpp}
\pagebreak

\inputminted[linenos, frame=lines]{haskell}{problemas/haskell/EC_WORLD.hs}
\pagebreak

\begin{figure}[h]
\centering
\includegraphics[width=\textwidth]{spoj/EC_WORLD-accepted-cpp}
\caption{El código fue aceptado por el juez en Haskell y C++}
\end{figure}

\subsection{Extender el palíndromo}
\includepdf[pages=-]{problemas/pdf/EPALIN.pdf}

\inputminted[linenos, frame=lines]{cpp}{problemas/cpp/EPALIN.cpp}
\pagebreak

\inputminted[linenos, frame=lines]{haskell}{problemas/haskell/EPALIN.hs}
\pagebreak

\subsection{Encontrar todas las ocurrencias de un patrón en un texto}
\includepdf[pages=-]{problemas/pdf/NHAY.pdf}

\inputminted[linenos, frame=lines]{cpp}{problemas/cpp/NHAY.cpp}
\pagebreak

\inputminted[linenos, frame=lines]{haskell}{problemas/haskell/NHAY.hs}
\pagebreak

% TODO: Chane y justifico esto el <$> https://ro-che.info/articles/2019-07-22-associativity-of-fmap



% TODO: poner que puedo optimizar la entrada con Data.ByteString.Char8