\documentclass[letterpaper,10pt]{article}
\usepackage[utf8]{inputenc}
\usepackage[spanish]{babel}
\usepackage[]{enumitem}
\usepackage[]{hyperref}
\pagestyle{empty}
\usepackage[left=2cm,right=2cm,top=1.2cm,bottom=1.2cm]{geometry}


\begin{document}
\noindent
\large
\textbf{Análisis algoritmo Knuth-Morris-Pratt puramente funcional} \\\\
\textbf{Ángel Iván Gladín García} \\
\normalsize Resumen de Proyecto de Tesis       \hfill No. cuenta: 313112470\\
\hfill 3 de noviembre de 2020.\\

\vspace*{-15pt}

\section{Resumen}
Los algoritmos de búsqueda de subcadenas son una clase de algoritmos de cadenas que tratan de buscar un(os)
\emph{patrones} en una cadena o texto. Normalmente cuando es una cadena grande se busca mejorar el desempeño del mismo.
El algoritmo de Knuth-Morris-Pratt es útil para buscar un patrón (con longitud $m$) en un texto (de longitud $n$) con una
complejidad en tiempo de $\Theta(n + m)$.

Se hará un analisis de versión puramente funcional así como su implementación, y finalmente se compará su desempeño con
la parte imperativa.

\section{Índice tentativo}

\begin{enumerate}

    \item Introducción
    \begin{itemize}
        \item Programación funcional
        \item Definicionees inductivas y recursivas
        \item Razonamiento ecuacional
        \item Definiciones de listas
        \item Principio de Fusión
    \end{itemize}

    \item Analisis de tiempo
    \begin{itemize}
        \item Notación asintótica
        \item Estimando tiempo
        \item Tiempo amortizado
    \end{itemize}

    \item String Matching (Algoritmo de búsqueda de subcadenas)
    \begin{itemize}
        \item Motivación
        \item Notación y terminología
        \item Algoritmo de búsqueda de subcadenas ingenuo (\textit{naïve})
        \item Diferentes tipos de algoritmos en cadenas
    \end{itemize}

    \item Análisis versión funcional
    \begin{itemize}
        \item Primer acercamiento
        \item Refinamiento de los datos
        \item Árboles
        \item Ejemplos
    \end{itemize}

    \item Algoritmo Knuth-Morris-Pratt
    \begin{itemize}
        \item Versión impertiva
        \item Versión funcional
    \end{itemize}
    
    \item \textit{Benchmarks}: versión imperativa contra versión funcional 

\end{enumerate}

\section{Bibliografía básica}

\begin{enumerate}[label={[\arabic*]}]
\item Bird, R. (2010). Pearls of Functional Algorithm Design. Cambridge: Cambridge University Press.
\item Bird, R., \& Gibbons, J. (2020). Algorithm Design with Haskell. Cambridge: Cambridge University Press.
\item Bird, R. (2014). Thinking Functionally with Haskell. Cambridge: Cambridge University Press.
\item Introduction to Algorithms T. Cormen, C. Leiserson, R. Rivest, and C. Stein. The MIT Press, 3rd Edition.
\end{enumerate}

\end{document}
