\documentclass[letterpaper,10pt]{article}
\usepackage[utf8]{inputenc}
\usepackage[spanish]{babel}
\usepackage[]{enumitem}
\usepackage[]{hyperref}
\pagestyle{empty}
\usepackage[left=2cm,right=2cm,top=1.2cm,bottom=1.2cm]{geometry}


\begin{document}
\noindent
\large
\textbf{Análisis del algoritmo Knuth-Morris-Pratt con énfasis en la programación funcional} \\\\
\textbf{Ángel Iván Gladín García} \\
\normalsize Resumen de Proyecto de Tesis       \hfill No. cuenta: 313112470\\
\hfill 8 de marzo de 2021.\\

\vspace*{-15pt}

\section{Resumen}
Los algoritmos de búsqueda de subcadenas son una clase de algoritmos de cadenas que tratan de
buscar un(os) \emph{patron(es)} en una cadena o texto. Normalmente cuando es una cadena grande se
busca mejorar el desempeño del mismo. El algoritmo de Knuth-Morris-Pratt (KMP) es útil para buscar un
solo patrón (con longitud $m$) en un texto (de longitud $n$) con una complejidad en tiempo de
$\Theta(n + m)$.

En este trabajo se hará énfasis en éste algoritmo centrado en la programación funcional; primero se
dará una especificación formal y por medio de razonamiento ecuacional se refinará una versión
más eficiente en complejidad en tiempo. En este análisis se desglosará la teoría necesaría para
poder llevar a cabo este proceso.

Finalmente se usará este algoritmo para resolver algunos problemas famosos que llegan a aparecer
en competencias o acertijos de programación competiviva y se discutirá su uso con sus respectivas
implicaciones.

\section{Índice tentativo}

\begin{enumerate}

\item \textbf{Fundamentos:} En esta sección se darán definiciones y teoremas necesarios usadas en
la programación funcional para poder profundizar en el razonamiento ecuacional. También se darán
propiedades y definiciones útiles respecto a \textit{cadenas}, ánalisis de tiempo en algoritmos y
un primer acercamiento a búsqueda de subcadenas.

\item \textbf{Algoritmos de subcadenas con perspectiva en \textit{la programación imperativa}:} Se
presentarán dos algoritmos; la versión ingenua y el algoritmo KMP y se analizarán algunos ejemplos
y el algoritmo \textit{per se}.

\item \textbf{Algoritmos de subcadenas con perspectiva en \textit{la programación funcional}:}
Aquí es donde se hará uso de todos los fundamentos presentados en el primer capítulo para hacer la
\textit{derivación} del algoritmo ingenuo al algoritmo Morris-Pratt y después a Knuth-Morris-Pratt.
También se discutirá sobre la \textit{función de error} que es usada en KMP. Se hará hincapié en
los teoremas y \textit{leyes} usadas en la programación funcional.

\item \textbf{\textit{Quickcheck}:} Se hablará acerca de esta biblioteca en Haskell para poder
probar las propiedades que deberían de cumplir las funciones, es decir, cada función tiene
propiedades deseables y usando QuickCheck se demuestra si se cumplen total o parcialmente
estas propiedades. Una ventaja notoria es que una propiedad es probada con una gran cantidad de
casos generados aleatoriamente.

%Será usado para probar las implentaciones derivadas con la implentación ingenua.

\item \textbf{Jueces en línea:} Se hablará brevemente en qué consiste la programación competitiva
y algunas ``heurísticas'' a seguir en la resolución de problemas. Al final se atacarán tres
problemas en los lenguajes de programación Haskell y C++ usando el algoritmo KMP y la función
de error.

\item \textbf{Conclusiones:} Se centrará en las ventajas y desventajas del razonamiento
ecuacional en la programación funcional, se hablará sobre algunas herramientas para poder
verificar nuestros programas y se darán ideas a trabajos futuros.

\item \textbf{Apéndices*:} Se abordarán cuestiones técnicas.

\end{enumerate}

\section{Bibliografía básica}

\begin{enumerate}[label={[\arabic*]}]
\item Bird, R. (2010). Pearls of Functional Algorithm Design.
Cambridge: Cambridge University Press.
\item Bird, R. (2014). Thinking Functionally with Haskell. Cambridge: Cambridge University Press.
\item Graham Hutton. ``A Tutorial on the Universality and Expressiveness of Fold''.
En: J. Funct. Program. 9.4 (jul. de 1999), págs. 355-372.
\url{https://doi.org/10.1017/S0956796899003500}.
\item Richard Bird. ``On building cyclic and shared structures in Haskell''. En: Formal Aspects of
Computing 24 (jul. de 2012).
\item Introduction to Algorithms T. Cormen, C. Leiserson, R. Rivest, and C. Stein. The MIT Press,
3rd Edition.
\end{enumerate}

\end{document}
