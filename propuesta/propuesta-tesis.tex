\documentclass[letterpaper,11pt]{article}
\usepackage[utf8]{inputenc}
\usepackage[spanish]{babel}
\usepackage[]{enumitem}
\usepackage[]{hyperref}
\pagestyle{empty}
\usepackage[left=2cm,right=2cm,top=1.2cm,bottom=1.2cm]{geometry}
\renewcommand{\labelenumii}{\theenumii}
\renewcommand{\theenumii}{\theenumi.\arabic{enumii}.}


\begin{document}
\noindent
\large
\textbf{Análisis del algoritmo Knuth-Morris-Pratt con énfasis en la programación funcional} \\\\
\textbf{Ángel Iván Gladín García} \\
\normalsize Resumen de Proyecto de Tesis       \hfill No. cuenta: 313112470\\
\hfill 24 de marzo de 2021.\\

\vspace*{-15pt}

\section{Resumen}
Los algoritmos de búsqueda de subcadenas son una clase de algoritmos de cadenas que tratan de
buscar un(os) \emph{patrón(es)} en una cadena o texto.
Una manera ingenua de hacerlo es mediante fuerza bruta; la idea es ir 'deslizando' el patrón sobre
el texto de izquierda a derecha, comparándolo con las subcadenas del mismo tamaño que empiezan en
cada carácter del texto. Este proceso, en el peor de los casos, toma una complejidad de tiempo de $O(nm)$.
% Normalmente cuando es una cadena grande se busca mejorar el desempeño del mismo. 
El algoritmo de Knuth-Morris-Pratt (KMP) es útil para buscar un solo patrón (con longitud $m$) en un texto (de
longitud $n$) con una complejidad en tiempo de $\Theta(n + m)$. 

En este trabajo se abordará el algoritmo KMP centrado en la programación funcional.
Primero se dará una especificación formal (en el lenguaje de programación \textsc{Haskell}) del algoritmo
de fuerza bruta con complejidad $O(nm)$ y por medio de razonamiento ecuacional se busca 
derivar el algoritmo KMP que mejora el desempeño del algoritmo anterior. % con complejidad $\Theta(n + m)$.

A diferencia de la implementación del algoritmo en una versión imperativa, usando un estilo funcional 
se podrá enfatizar la idea principal del algoritmo al razonar en alto nivel y obtener 
de una manera más \textit{elegante} una implementación funcional de este. 
En ambas versiones, la imperativa y la funcional, el algoritmo KMP tiene la misma 
complejidad: $\Theta(n + m)$.

Para el análisis anterior se desglosará la teoría necesaría para poder llevar a cabo el proceso en la
programación funcional. También se hablará sobre una biblioteca en \textsc{Haskell} llamada 
\textsc{QuickCheck} para probar ambas versiones con casos generados aleatoriamente.

Finalmente se explicará brevemente en qué consiste la programación competitiva y los jueces en línea.
Este tema es parte de la motivación del trabajo al retomar algunos problemas propuestos para estos concursos.
En esta parte, usando el algoritmo KMP y la función de error (función auxiliar usada en KMP) se resolverán e
implementarán algunos problemas populares como encontrar el factor de repetición de una cadena,
decidir si una cadena es rotación cíclica de otra y cómo extender una cadena a un palíndromo de
longitud mínima. Se implementarán tanto en Haskell como en C++ y se harán sus respectivas
comparaciones.

\section{Índice tentativo}

\begin{enumerate}

\item \textbf{Fundamentos:} En esta sección se darán definiciones y teoremas necesarios usadas en
la programación funcional para poder profundizar en el razonamiento ecuacional. También se darán
propiedades y definiciones útiles respecto a la teoría de \textit{cadenas}, 
al ánalisis de tiempo en algoritmos y un primer acercamiento a la búsqueda de subcadenas.
\begin{enumerate}
    \item Programación funcional
    \item Listas, plieges, principio universal de fusión, \textit{Scan Lemma}, \textit{Tuppling},
    \textit{Accumulating Parameters}
    \item Razonamiento ecuacional
    \item Notación asintótica
    \item Cadenas
\end{enumerate}

\item \textbf{Algoritmos de búsqueda de subcadenas con perspectiva en \textit{la programación
imperativa}:} Se presentarán dos algoritmos; la versión ingenua y el algoritmo KMP y se analizarán
algunos ejemplos y el algoritmo \textit{per se}.
\begin{enumerate}
    \item Algoritmo de búqueda de subcadenas ingenuo (naïve)
    \item Función de error
    \item De Morris-Pratt a Knuth-Morris-Pratt
    \item Implementación en C++
\end{enumerate}

\item \textbf{Algoritmos de búsqueda de subcadenas con perspectiva en \textit{la programación
funcional}:} En esta parte se hará uso de todos los fundamentos presentados en el primer capítulo
para hacer una \textit{derivación} desde el algoritmo ingenuo hacia el algoritmo Morris-Pratt y 
después extenderlo al algoritmo Knuth-Morris-Pratt. 
También se discutirá sobre la \textit{función de error}, función auxiliar que es usada en KMP. 
Se hará hincapié en los teoremas y \textit{leyes} usadas en la programación funcional.
\begin{enumerate}
    \item Derivando Función de error
    \item Derivando Knuth-Morris-Pratt
\end{enumerate}

\item \textbf{\textit{QuickCheck}:} Se hablará acerca de esta biblioteca en Haskell para poder
probar las propiedades que deben cumplir las funciones, es decir, cada función tiene
propiedades deseables y usando QuickCheck se demuestra si se cumplen total o parcialmente
estas propiedades. Una ventaja notoria de esta herramienta es que asegura que si una propiedad 
es probada con una gran cantidad de casos generados aleatoriamente se cumplen dichas propiedades.
\begin{enumerate}
    \item Motivación
    \item Uso básico
    \item Generadores
    \item Probando las implementaciones
\end{enumerate}

\item \textbf{Jueces en línea:} Se hablará brevemente en qué consiste la programación competitiva
y algunas ``heurísticas'' a seguir en la resolución de problemas. Al final se atacarán tres
problemas en los lenguajes de programación Haskell y C++ usando el algoritmo KMP y la función
de error.
\begin{enumerate}
    \item Diferentes jueces en línea
    \item Problemas: Encontrar el factor de repetición de una cadena, ver si una cadena es una
    rotación cíclica de otra, extender una cadena a un palíndromo de longitud mínima.
\end{enumerate}

\item \textbf{Conclusiones:} Se centrará en las ventajas y desventajas del razonamiento
ecuacional en la programación funcional, se hablará sobre algunas herramientas para poder
verificar nuestros programas y se darán ideas a trabajos futuros.

%\item \textbf{Apéndices*:} Se abordarán cuestiones técnicas.

\end{enumerate}

\section{Bibliografía básica}

\begin{enumerate}[label={[\arabic*]}]
\item Richard Bird y Jeremy Gibbons. Algorithm Design with Haskell. Cambridge University Press,
2020. doi: 10.1017/9781108869041.
\item Thomas H. Cormen y col. Introduction to Algorithms, Third Edition. 3rd. The MIT Press,
2009. isbn: 0262033844.
\item Graham Hutton. Programming in Haskell. 2nd. USA: Cambridge University Press, 2016.
isbn: 1316626229.
\item Wojciech Rytter Maxime Crochemore. Jewels of stringology. 1st. World Scienti- fic, 2002.
isbn: 9789810247829,9789812778222,9810247826.
\item QuickCheck: An Automatic Testing Tool for Haskell.
\url{http://www.cse.chalmers. se/~rjmh/QuickCheck/manual.html.}
\item Richard Bird. ``On building cyclic and shared structures in Haskell''. En: Formal Aspects of
Computing 24 (jul. de 2012).
\item Graham Hutton. ``A Tutorial on the Universality and Expressiveness of Fold''.
En: J. Funct. Program. 9.4 (jul. de 1999), págs. 355-372.
\url{https://doi.org/10.1017/S0956796899003500}.

\end{enumerate}

\end{document}
