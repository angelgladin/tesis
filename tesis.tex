\documentclass{book}

\usepackage{defs-config-macros}


\begin{document}

\thispagestyle{empty}
\frontmatter
    \begin{minipage}{.3\textwidth}
  \flushleft
  \center{\includegraphics[scale=.09]{unam.pdf}}

  \vspace{20pt}

  \center{
    \rule{.5pt}{.6\textheight}
    \hspace{7pt}
    \rule{2pt}{.6\textheight}
    \hspace{7pt}
    \rule{.5pt}{.6\textheight}
  } \\

  \center{\includegraphics[scale=.22]{ciencias.pdf}}
\end{minipage}
\begin{minipage}{.7\textwidth}
\flushright

\center{

  \center{
    \LARGE{U}\large{NIVERSIDAD} \LARGE{N}\large{ACIONAL}
    \LARGE{A}\large{UTÓNOMA} \\[10pt]
    \large{DE}
    \LARGE{M}\large{ÉXICO}
  } \\
  \rule{\textwidth}{2pt}
  \\
  \hrulefill\\[1cm]

  \LARGE{F}\large{ACULTAD DE } \LARGE{C}\large{IENCIAS}\\[2cm]

  \large{
  Análisis del algoritmo \textit{Knuth-Morris-Pratt} con énfasis en la programación funcional
  }\\[1.6cm]

  \huge{
T \hspace{1cm} E \hspace{1cm} S \hspace{1cm} I \hspace{1cm} S  }\\[1cm]

  \large{QUE PARA OBTENER EL TÍTULO DE:}\\[1cm]

  \large{
Licenciado en Ciencias de la Computación  }\\[1cm]

  \large{PRESENTA:}\\[1cm]

  \large{
Ángel Iván Gladín García  }\\[1cm]

  \large{
TUTORA  }\\[.2cm]

  \large{
  Dra. Lourdes Del Carmen Gonzales Huesca}\\[1cm]
  \large{
    Ciudad Universitaria, Cd. Mx., 2021
  }
}

\end{minipage}

    \clearpage
    \mbox{}
    \clearpage
    \thispagestyle{empty}
    
    \pagenumbering{Roman} 


\begin{center}
{\Large \textbf{Hoja de Datos del Jurado}}
\vspace*{.85cm}
\end{center}


\begin{enumerate}

\item Datos del Alumno

Gladín \\
García \\
Ángel Iván \\
+52 55 8196 8560 \\
Universidad Nacional Autónoma de México \\
Facultad de Ciencias \\
Ciencias de la Computación \\
313112470


\item Datos de la Tutora

Dra. \\
Lourdes del Carmen \\
González \\
Huesca


\item Datos del Sinodal 1

Dr. \\
Favio Ezequiel \\
Miranda \\
Perea


\item Datos del Sinodal 2

Dra. \\
Adriana \\
Ramírez \\
Vigueras


\item Datos del Sinodal 3

Dr. \\
Canek \\
Peláez \\
Valdés


\item Datos del Sinodal 4

L. en C.C. \\
Fernando Abigail \\
Galicia \\
Mendoza


\item Datos del trabajo escrito

{\small Análisis del algoritmo \textit{Knuth-Morris-Pratt} con énfasis en la programación funcional} \\
100p. \\ % TODO: cambiar el número de hojas
2021 


\end{enumerate} 

    
    \chapter*{Agradecimientos}
    \input{capitulos/agradecimientos}
    \clearpage
    
    \tableofcontents


\mainmatter
    \chapter*{Motivación y estructura del trabajo}
        En mi camino aprendiendo y escribiendo programas usando programación funcional, es algo común ver un programa
que aunque sea corto y legible, muchas veces es algo ineficiente. Entonces es ahí cuando Richard Bird tiene
en mente que un programa debe actuar como la especificación formal del problema, pero también por medio del
razonamiento ecuacional poder calcular uno más eficiente.
Uno de los factores que ayudó al crecimiento en el interés de la programación funcional, fue que en los años
1990's se dieron cuenta que estos lenguajes son buenos para hacer razonamiento ecuacional.
De hecho el lenguaje funcional Gofer, inventado por Mark Jones capturó este pensamiento como un acrónimo 
(\textit{Good for equational reasoning}).
\newline

Lo que se abordará en este trabajo es primero empezar con una especificación en Haskell y después proseguir a
calcular una versión más eficiente por medio de razonamiento ecuacional.
La razón de este trabajo es ver hasta donde el diseño de un algoritmo puede estar encajado en una forma
matemática de calcular un resultando usando principios matemáticos bien establedicos como definiciones, 
teoremas, y \textit{``leyes''}.
Curiosamente, es generalmente verdadero que en matemáticas los cálculos están diseñados para simplificar
cosas complicadas, en el diseño de algoritmos usualmente es al revés.
\begin{quote}
Simples, pero ineficientes programas son transformados en versiones más eficientes que puedes ser
completamente opacas en su implementación.
\end{quote}
Explicando las ideas detrás de un algoritmo es mucho más fácil en un estilo funcional, en vez de un
procedimental. Las funciones pueden ser separadas fácilemente, cada una es sucinta y capturan patrones
de cómputo.
\newline

Los algoritmos de búsqueda de subcadenas (\textit{String Matching Algorithms}) son usados frecuentemente en:
programas de edición de texto para encontrar todas las ocurrencias de un patrón en un texto, para encontrar
patrones particulares en una cadenas de ADN, o también en algunos motores de búsqueda los utilizan para
encontrar páginas web en búsquedas, entre otras aplicaciones. Algoritmos efecientes para atacar este tipo de
problemas nos ayudan gratamente para mejorar el tiempo de la búsqueda.
\newline

TODO

Como lo menciona Richard S. Bird en su artículo \emph{Polymorphic String Matching}\cite{book:1505279}
% TODO: ponerle un formato bonito
El desarrollo de cálculos en programas funcionales ha sido asociado a trucos de magia: agradables de ver pero seguido pero a menudo hay un misterio en cómo se hacen.

En este trabajo se explicará esto, es dar un cálculo del algoritmo KMP,
%FIXME: quitar lo de abajo?
Este probleme de string matching está formulado polimórficamente, así que la única propiedad disponible que tienen los elementos del alfabeto es que sean comparables.

% TODO: de aquí me puedo sacar algunas ideas
% https://www.cs.princeton.edu/~rs/AlgsDS07/21PatternMatching.pdf

% TODO: poner que así organizo los .bib
% https://flamingtempura.github.io/bibtex-tidy/
    \addcontentsline{toc}{chapter}{Motivación y estructura del trabajo}
    
    \chapter{Fundamentos}
        En este capítulo se verá como, dado un conjunto de definiciones de funciones, por medio de razonamiento ecuacional podemos llegar a otras definiciones y/o probarlas. Las pruebas aquí se harán mediantte inducción.

Muchas veces es algo engorroso probar funciones similares repetidamente, por eso veremos una forma de hacer pruebas (en algunos casos) más cortas,
presentando unas \textit{funciones de orden superior} que encapsulan patrones comunes de cómputo. Y así, probar resultados más generales y apelar a ellos.

Al final se verá que la eficiencia también importa, porque se mostrarán algunos ejemplos; como un problema famoso llamado \textit{``The maximum segment sum''}
y una mejora de la función \texttt{scanr}. Y todo esto se logrará como consecuencia de lo dicho anteriormente.

\subsection{Inducción sobre listas}
Recordemos que toda lista finita es de la forma; una lista vacía \texttt{[]} ó \texttt{x:xs} donde \texttt{xs} es una lista finita. Por consiguiente, para probar que $P(xs)$ se mantiene para todas las lista finitas $xs$,
se tiene que probar que:

\begin{enumerate}
    \item $P([])$ se cumple
    \item Para toda $x$ y para todas las listas finitas $xs$, que $P(x:xs)$ se cumple dado que $P(xs)$ también.
\end{enumerate}

Tomemos la definición de concatenación \texttt{(++)},
\inputminted{haskell}{definiciones/concatenation.hs}

Y ahora probemos que (++) es asociativa para todas las listas finitas $xs$, es decir:

(xs ++ ys) ++ zs = xs ++ (ys ++ zs)

Por inducción sobre $xs$:
% TODO: poner aquí lo del strictness property https://stackoverflow.com/questions/27672585/efficient-version-of-inits/27674051#27674051

%{\displaystyle x+5} is the left-hand side (LHS) and {\displaystyle y+8}{\displaystyle y+8} is the right-hand side (RHS).
%FIXME: si caben dos eucaciones por lado lo hago en columnas.

\begin{itemize}
\item Caso []
\begin{minted}{haskell}
(LHS)

([] ++ ys) ++ zs
=   {++.1}
ys ++ zs
\end{minted}

\begin{minted}{haskell}
(RHS)

[] ++ (ys ++ zs)
=   {++.1}
ys ++ zs
\end{minted}

\item Caso (x:xs)
\begin{minted}{haskell}
(LHS)

((x:xs) ++ ys) ++ zs
=   {++.2}
(x:(xs ++ ys)) ++ zs
=   {++.2}
x:((xs ++ ys) ++ zs)
\end{minted}

\begin{minted}{haskell}
(RHS)

(x:xs) ++ (ys ++ zs)
=   {++.2}
x:(xs ++ (ys ++ zs))
=   {induction}
x:((xs ++ ys) ++ zs)
\end{minted}

\end{itemize}

\section{Pliegues}

\subsection{\texttt{foldr}}
\inputminted{haskell}{definiciones/foldr.hs}

\subsection{Síntesis de programas vía la propiedad universal}


\subsection{\texttt{foldl}}
\inputminted{haskell}{definiciones/foldl.hs}



\section{Programación funcional}
De forma muy general y resumiendo, la programación funcional:
\begin{itemize}
    \item es un método de construcción de un programa que hace énfasis en las funciones y sus aplicaciones
    en vez de cómandos y sus ejecuciones.

    \item usa notación matemática simple que permite que los problemas sean descritos de manera clara
    y consisa.
    \item tiene bases matemáticas que fundamentan el razonamiento ecuacional acerca de las propiedades de
    los programas.
\end{itemize}

\section{Definiciones inductivas y recursivas}
%TODO

\section{Razonamiento ecuacional}
%TODO


\section{Definiciones de listas}
%TODO

\section{Definiciones de funciones}
\inputminted{haskell}{definiciones/map.hs}



\section{Ley de Fusión}

\begin{minted}{haskell}
f . foldr g a = foldr h b
\end{minted}

\begin{itemize}
    \item $f$ es una función estricta.
    \item $f a = b$
    \item $f (g y x) = h (f y) x$ para toda $x$ y $y$.w
\end{itemize}

Ejmeplos:
\begin{itemize}
    \item \hsCode{double . sum    = foldr ((+) . double) 0}
    \item \hsCode{length . concat = foldr ((+) . length) 0}
\end{itemize}

% TODO poner lo de una función estrictca, está en el thinking functionally with hskell página 29

%% TODO: poner aquí lo de lados de la ecuación
% https://en.wikipedia.org/wiki/Sides_of_an_equation

% Polymorphic algorithms
% https://wiki.haskell.org/Polymorphism#:~:text=A%20value%20is%20polymorphic%20if,polymorphism%20and%20ad%2Dhoc%20polymorphism

% foldr
% https://wiki.haskell.org/Foldr_Foldl_Foldl'
%http://www.cantab.net/users/antoni.diller/haskell/units/unit06.html

\section{\textit{Scan Lemma}}

En esta sección se considerará la función \hsCode{scanl}, \hsCode{scanl} aplica un pliegue
izquierdo (\hsCode{foldl f e}) a cada segmento (todos los prefijos) de una lista como se muestra
a continuación,

\begin{minted}{haskell}
scanl (@) e [x, y, z, ...] = [e, e@x,(e@x)@y,((e@x)@y)@z,...]
\end{minted}

Se propondrá la siguiente especificación de \hsCode{scanl} como:
\begin{minted}{haskell}
scanl :: (b -> a -> b) -> b -> [a] -> [b]
scanl f e = map (foldl f e) . inits

inits :: [a] -> [[a]]
inits []     = [[]]
inits (x:xs) = [] : map (x:) (inits xs)
\end{minted}

Ejemplo:
\begin{minted}{haskell}
>>> scanl (+) 0 [1..10]
[0,1,3,6,10,15,21,28,36,45,55]
\end{minted}

La expresión anterior calcula la suma de cada prefijo de una lista de números del 1 al 10.
\begin{minted}{haskell}
[0, 0+1, (0+1)+2, ((0+1)+2)+3, (((0+1)+2)+3)+4, ...]
\end{minted}

Veamos un ejemplo de \hsCode{scanl}
\begin{minted}{haskell}
>>> inits [1..5]
[[],[1],[1,2],[1,2,3],[1,2,3,4],[1,2,3,4,5]]
\end{minted}

Pero se puede ver que la definición propuesta de \hsCode{scanl} involucra evaluar \hsCode{f} un
total de $\sum_{i=0}^{n} i = \frac{n(n+1)}{2}$ veces sobre una lista de longitud $n$.

Es aquí cuando uno se pregunta si ¿se podría hacer mejor?. La respuesta es sí, cálculando una mejor
definiciónpor medio de razonamiento ecuacional. Cosideremos por casos.

\begin{itemize}
\item Caso \hsCode{[]}
\begin{minted}{haskell}
scanl f e []
  = -- {Definición de scanl}
map (foldl f e) (inits [])
  = -- {Por inits.1}
map (foldl f e) [[]]
  = -- {Por map.1 y map.2}
(foldl f e []) : map (foldl f e) []
  = -- {Por foldl.1 y map.1}
e : []
  = -- {Azucar sintáctica}
[e]
\end{minted}

Teniendo así que \hsCode{scanl f e [] = [e]}.

\item Caso \hsCode{x:xs}
\begin{minted}{haskell}
scanl f e (x:xs)
  = -- {Definición de scanl}
map (foldl f e) (inits (x:xs))
  = -- {Por inits.2}
map (foldl f e) ([] : map (x:) (inits xs))
  = -- {Por map.1 y map.2}
(foldl f e []) : (map (foldl f e . (x:)) (inits xs))
  = -- {Por foldl.1}
e : (map (foldl f e . (x:)) (inits xs))
  = -- {Por la afrimación que se demostrará abajo y que es el caso (x:xs)}
e : map (foldl f (f e x)) (inits xs)
  = -- {Por la primera definición de scanl.1}
e:scanl f (f e x)
\end{minted}

\textbf{Afirmación:} \hsCode{foldl f e . (x:) = foldl f (f e x)}. Se seguirá como una consecuencia
inmediata de \hsCode{foldl}.

\begin{itemize}
\item Caso \hsCode{x:[]}
\begin{minted}{haskell}
foldl f e . (x:) []
  = -- {Composición de funciones}
foldl f e [x]
  = -- {Aplicación y azucar sintáctica}
foldl f (f e x) []
  = -- {Por foldl.1 y Por foldl.2}
e
\end{minted}

\item Caso \hsCode{x:ys}, análogo al anterior.
\end{itemize}
\end{itemize}

Teniendo así una nueva definición de \hsCode{scanl} como,
\begin{minted}{haskell}
scanl f e []     = [e]
scanl f e (x:xs) = e:scanl f (f e x) xs
\end{minted}

donde \hsCode{f} solo se calcula un número lineal de veces, a diferencia de su primera definición
propuesta que requería un número cuadrático de veces.

Aunque si vemos definición\footnote{
\url{https://hackage.haskell.org/package/base-4.14.1.0/docs/src/GHC.List.html\#scanl}
} del preludio es diferente:
\begin{minted}{haskell}
scanl                   :: (b -> a -> b) -> b -> [a] -> [b]
scanl                   = scanlGo
  where
    scanlGo           :: (b -> a -> b) -> b -> [a] -> [b]
    scanlGo f q ls    = q : (case ls of
                               []   -> []
                               x:xs -> scanlGo f (f q x) xs)
\end{minted}

Pero esto de debe a que la versión que se calculó da \hsCode{scanl f e undefined = undefined} y la
versión de preludio \hsCode{scanl f e undefined = e:undefined}. Esto se debe a que como Haskell
es perezoso, no nos debemos de preguntar nada acerca de la lista a procesar, pero lo que es seguro
es que empieza con \texttt{e}.

En general, cualquier problema que involucre la función \hsCode{inits}, este lema es bastante útil
de saber porque si recordamos la primera especificación:
\begin{minted}{haskell}
scanl f e = map (foldl f e) . inits
\end{minted}

La LHS toma $\Theta(n)$ el número de evaluaciones de \hsCode{f} mientras que RHS toma $\Theta(n^2)$.
Y como se demostró ambas expresiones son equivalentes.

\begin{figure}[h]
\caption{Un ejemplo concreto de la versión propuesta inicialmente de \texttt{scanl} y la que se derivó}
\centering
\includegraphics[width=0.9\textwidth]{scan_lemma_example.pdf}
\end{figure}

\subsection{\textit{The maximum segment sum}}

\section{\textit{Tupling}}
\section{\textit{Strict property}}


% TODO: aquí agarrar cosas del polymorphic en la parte que dice: on compositioin https://wiki.haskell.org/Tutorials/Programming_Haskell/String_IO
    
    % TODO: Poner en dos capítulos separados lo de "Algoritmos de búsqueda de subcadenas" y función de error
    % unos imperativo y otro funcional
    \chapter{Algoritmos de búsqueda de subcadenas}
        \lipsum[1-1]

\section{Motivación}
\lipsum[2-4]

\section{Notación y terminología}
\lipsum[2-4]

\section{Algoritmo de búsqueda de subcadenas ingenuo (\textit{naïve})}
\lipsum[2-4]

\section{Diferentes tipos de algoritmos en cadenas}
\lipsum[2-4]


    \chapter{Función de error}
        En este capítulo se derivará por medio de una especificación formal la función de error del
algoritmo de KMP. Aunque por medio de este acercamiento\cite{bird:cyclic} se podría obtener todo
el algoritmo KMP usando estructuras cíclicas, específicameente listas doblemente ligadas, no se
hará así, porque se verá otra manera ``puramente funcional'' de hacerlo en el siguiente capítulo.

Consideremos la cadena \texttt{abacabab} y su procesamiento con la función de error,

\begin{table}[h]
\centering
\begin{tabular}{c|c|c|c|c|c|c|c|c|}
\cline{2-9}
$k$      & 1          & 2          & 3          & 4          & 5          & 6          & 7          & 8          \\ \hline
$P[k]$   & \texttt{a} & \texttt{b} & \texttt{a} & \texttt{c} & \texttt{a} & \texttt{b} & \texttt{a} & \texttt{b} \\ \hline
$\pi[k]$ & 0          & 0          & 1          & 0          & 1          & 2          & 3          & 2          \\ \cline{2-9} 
\end{tabular}
\end{table}

La entrada $\pi[k]$ en la posición $b$ es la longitud del \textbf{sufijo propio más largo} de
\hsCode{take k "abacabab"} que también es un prefijo de \hsCode{"abacabab"}.

Veamos del procesamiento de la cadena de arriba:
\begin{itemize}
\item[$\pi{[1]}$] Dado que el sufijo propio más largo de \hsCode{take 1 "abacabab" = "a"} es
$\varepsilon$, siempre la primera posición será 0. Por lo que $\pi[1] = 0$.
\item[$\pi{[2]}$] El sufijo propio más largo de \hsCode{take 2 "abacabab" = "ab"} que también es
prefijo de \texttt{abacabab} es $\varepsilon$. Por lo que $\pi[2] = 0$.
\item[$\pi{[3]}$] El sufijo propio más largo de \hsCode{take 3 "abacabab" = "aba"} que también es
prefijo de \texttt{abacabab} es \texttt{a}. Por lo que $\pi[3] = 1$.
\item[$\pi{[4]}$] El sufijo propio más largo de \hsCode{take 4 "abacabab" = "abac"} que también es
prefijo de \texttt{abacabab} es $\varepsilon$. Por lo que $\pi[4] = 0$.
\item[$\pi{[5]}$] El sufijo propio más largo de \hsCode{take 5 "abacabab" = "abaca"} que también
es prefijo de \texttt{abacabab} es \texttt{a}. Por lo que $\pi[5] = 1$.
\item[$\pi{[6]}$] El sufijo propio más largo de \hsCode{take 6 "abacabab" = "abacab"} que también
es prefijo de \texttt{abacabab} es \texttt{ab}. Por lo que $\pi[6] = 2$.
\item[$\pi{[7]}$] El sufijo propio más largo de \hsCode{take 7 "abacabab" = "abacaba"} que también
es prefijo de \texttt{abacabab} es \texttt{aba}. Por lo que $\pi[7] = 3$.
\item[$\pi{[8]}$] El sufijo propio más largo de \hsCode{take 8 "abacabab" = "abacabab"} que también
es prefijo de \texttt{abacabab} es \texttt{ab}. Por lo que $\pi[8] = 2$.
\end{itemize}

Para una lista no vacía \texttt{as}, los sufijos \textbf{propios} de \hsCode{take k as} son sufijos
de \hsCode{take (k-1) (tail as)}, el $k-1$ es porque como se afirma que la lista \texttt{as} es no
vacía, se contemplan en los sufijos propios a la lista vacía, y a \hsCode{tail as} de manera
similar.

Teniendo así una forma de calcular todos los sufijos \textbf{propios} de una lista \texttt{as} como:
\begin{minted}{haskell}
[take (k - 1) (tail as) | k <- [1 .. length as]] = inits (tail as)
\end{minted}

De igual manera se mostrará por medio de un ejemplo concreto tales sufijos propios,
\begin{minted}{haskell}
>>> inits (tail "abacabab")
    inits "bacabab"
    ["","b","ba","bac","baca","bacab","bacaba","bacabab"]
\end{minted}

donde \hsCode{inits}\footnote{
    La función \hsCode{inits} viene definida en el módulo \hsCode{Data.List}.
    }
regresa una lista con los prefijos de una lista en orden creciente y \hsCode{tail} extrae la cabeza
de la lista.

La función de error se define como:
\inputminted[fontsize=\small]{haskell}{codigo/haskell/FailureFunctionNaive.hs}

Donde \hsCode{llasp as bs} es la longitud del sufijo más largo de \texttt{bs} que es también un
prefijo de \texttt{as}. Esto es porque la función \hsCode{tails}\footnote{
    La función \hsCode{tails} viene definida en el módulo \hsCode{Data.List}.
    }
regresa todos los sufijos en orden decreciente y con la función \hsCode{head} solo se obtiene el
primero (que evidentemente es el mayor). Y así \hsCode{ptable} calcula la función de error como
una lista.

Si se calcula \hsCode{ptable "abacabab"} el resultado es:
\begin{minted}{haskell}
>>> ptable "abacabab"
    [(1,('a',0)),(2,('b',0)),(3,('a',1)),(4,('c',0)),(5,('a',1)),
     (6,('b',2)),(7,('a',3)),(8,('b',2))]
\end{minted}

Recordando el \textit{Scan Lemma}\ref{fundamentos:scan_lemma} y viendo que hay un \hsCode{inits} en
\hsCode{ptable} veamos si se puede optimizar. El \textit{Scan Lemma} afirma que
\begin{minted}{haskell}
map (foldl op e) . inits = scanl op e
\end{minted}

Para expresar a \hsCode{llasp as bs} en la forma de un pliegue \hsCode{foldl (op as) e} se debe
demostrar que: % TODO: The universal property of fold

\begin{minted}{haskell}
llsap as []          = e
llsap as (bs ++ [b]) = op as (llsap as bs) b
\end{minted}
Para una definición adecuada de \hsCode{e} y \hsCode{op}. 

Es inmediato que \hsCode{llsap as [] = 0} porque el sufijo \texttt{[]} que es prefijo de
\texttt{as} es de longitud 0 y así \texttt{e = 0}.

Lo interesante es encontrar \hsCode{op}. Sea\hsCode{ k = llsap as bs} donde \texttt{k} y\\
\hsCode{ a = head (drop k as)}, así \texttt{a} es el siguiente elemento de \texttt{as} después del
prefijo más largo de \texttt{as} que se empareja con el sufijo de \texttt{bs}.

Para ejemplificarlo consideremos los siguientes ejemplos sobre la cadena original \texttt{abacabab}.
\begin{itemize}
\item Consideremos $i = 3$ donde\hsCode{ bs = take 3 "abacabab" = "aba"},\\
\hsCode{llsap "abacabab" "a" = 1}\\
\hsCode{head (drop 1 "abacabab") = head "bacabab" = 'b' = a}.
Y así \texttt{a\colorbox{yellow}b}\texttt{acabab}.
\item Consideremos $i = 6$ donde\hsCode{ bs = take 6 "abacabab" = "abacab"},\\
\hsCode{llsap "abacabab" "ab" = 2}\\
\hsCode{head (drop 2 "abacabab") = head "acabab" = 'a' = a}.
Y así \texttt{ab\colorbox{yellow}a}\texttt{cabab}.
\end{itemize}

Si \texttt{a = b}, entonces \hsCode{llsap as (bs ++ [b]) = k+1} porque % TODO: explicar

Si no, \texttt{k = 0} entonces \hsCode{llsap as (bs ++ [b]) = 0} porque % TODO: porque no hay sufifo que sea prefijo

En el caso que queda se tiene % TODO: explicar que es porque, pero en sí es porque es un segmento de bs es un segmento de take (k-1) as
\begin{minted}{haskell}
llsap as (bs ++ [b]) = llsap as (take (k - 1) (tail as) ++ [b])
\end{minted}

Así \texttt{llasp as bs = foldl (op as) bs}, donde
\begin{minted}{haskell}
op as k b  | a == b    = k + 1
           | k == 0    = 0
           | otherwise = llsap as (take (k - 1) (tail as) ++ [b])
               where a = head (drop k as)
\end{minted}

% TODO: quitar esto de los backpointers, es más ni lo usi
Ahora viene la parte donde la representación de la tabla de apuntadores \textit{back} como un arreglo en vez de una lista
resulta útil. Poniendo \texttt{xa = ptable as}, se puede ver que,

% TODO: decir que aquí usaré ya arrays
\begin{minted}{haskell}
head (drop k as)                        = fst (xa!(k + 1))
llsap as (take (k - 1) (tail as) ++ [b] = op as (llsap as (take (k - 1) (tail as))) b
                                        = op as (snd (xa!k)) b
\end{minted}

% TODO: poner paso intermeedio de lista a arreglo.
Recordando que \hsCode{xa} es 1-indexada. Esto significa que \hsCode{ptable} queda redefinida como


\inputminted{haskell}{codigo/haskell/FailureFunctionOptimized.hs}


    \chapter{Algoritmo Knuth-Morris-Pratt}
        \lipsum[1-1]

\section{Versión imperativa}
\inputminted{cpp}{codigo/cpp/kmp.cpp}


\section{Versión funcional}
\inputminted{haskell}{codigo/haskell/3-trees.hs}


    \chapter{Entrada y Salida perezosa (Lazy I/O)}
        As the Camel Book says:
\epigraph{I recall seeing a package to make quotes}{Snowball}

Unless you're using artificial intelligence to model a solipsistic philosopher, your program needs some way to communicate with the outside world.



% TODO: Poner las funciones que uttilicé para procesar el input
% lines y esas https://hackage.haskell.org/package/base-4.14.0.0/docs/Data-List.html 


% TODO: explicar I/O haskell
% https://byorgey.wordpress.com/2019/05/22/competitive-programming-in-haskell-scanner/
% https://byorgey.wordpress.com/2019/04/24/competitive-programming-in-haskell-basic-setup/

% https://wiki.haskell.org/Tutorials/Programming_Haskell/String_IO
% 

    \chapter{QuickCheck}
        %import qualified Test.QuickCheck as QC

    \chapter{Jueces en línea}
        \includepdf[pages=-]{problemas/pdf/FINDSR.pdf}
\includepdf[pages=-]{problemas/pdf/EPALIN.pdf}
\includepdf[pages=-]{problemas/pdf/NHAY.pdf}

% TODO: Chane y justifico esto el <$> https://ro-che.info/articles/2019-07-22-associativity-of-fmap
% https://wiki.haskell.org/SPOJ
    
    \chapter{Conclusiones}
        \input{capitulos/11-conclusiones}
    
    \begin{appendices}
        %Para darle formato al código en Haskell (dado que hacerlo de forma manual) es una tanto tedioso se utilizó
%\href{https://github.com/jaspervdj/stylish-haskell}{\texttt{stylish-haskell}}.

%A la hora de buscar ``símbolos'' en de manera más inteligente en \LaTeX se utilizó
%\href{https://github.com/jaspervdj/stylish-haskell}{Detexify}, haciendo un ezbozo del símbolo
%a buscar y nos dará sugerencias de cuál podría ser.


%De igual manera el código en C++ se le dio estiló uttilizando el estándar de \href{https://llvm.org/docs/CodingStandards.html}{LLVM}

%Lazy I/O

%# Explicar la entrada en haskell

%Ver si cambio `fmap` con `<$>`

%- `lines`,
%- Para leer cada test case use una función auxiliar llamada `join`

\section{Compilar y ejecutar código}


As the Camel Book says:
\epigraph{I recall seeing a package to make quotes}{Snowball}

Unless you're using artificial intelligence to model a solipsistic philosopher, your program needs some way to communicate with the outside world.



% TODO: Poner las funciones que uttilicé para procesar el input
% lines y esas https://hackage.haskell.org/package/base-4.14.0.0/docs/Data-List.html 


% TODO: explicar I/O haskell
% https://byorgey.wordpress.com/2019/05/22/competitive-programming-in-haskell-scanner/
% https://byorgey.wordpress.com/2019/04/24/competitive-programming-in-haskell-basic-setup/

% https://wiki.haskell.org/Tutorials/Programming_Haskell/String_IO
% 
        
        % TODO: poner que así organizo los .bib
% https://flamingtempura.github.io/bibtex-tidy/            
    \end{appendices}

\backmatter
    \nocite{*}
    \printbibliography

\end{document}
