\documentclass{book}

\usepackage[utf8]{inputenc}
\usepackage[spanish]{babel}
\usepackage{geometry} % Márgenes en las páginas
\usepackage[backend=biber]{biblatex} % Poder agregar la bibliografia con un .bib
\usepackage{graphicx} % Para poder agregar las imágenes con y usar \graphicspath
\usepackage{lipsum} % Usar texto dummy para llenar hojas

\usepackage[ruled, vlined, linesnumbered]{algorithm2e} % Incluir pseudocódigo

%\usepackage{amsmath}
%\usepackage{amssymb}
%\usepackage[normalem]{ulem}
%\usepackage{hyperref}
%\usepackage{float}
%\usepackage[Sonny]{fncychap}
%\usepackage{algorithm}
%\usepackage{algpseudocode}
%\usepackage[table,xcdraw]{xcolor}
%\usepackage[export]{adjustbox}
%\usepackage[all]{nowidow}


\geometry{bindingoffset=3cm}

\addbibresource{referencias.bib}
\graphicspath{{imagenes/}}

\begin{document}

\thispagestyle{empty}
\frontmatter
    \begin{minipage}{.3\textwidth}
  \flushleft
  \center{\includegraphics[scale=.09]{unam.pdf}}

  \vspace{20pt}

  \center{
    \rule{.5pt}{.6\textheight}
    \hspace{7pt}
    \rule{2pt}{.6\textheight}
    \hspace{7pt}
    \rule{.5pt}{.6\textheight}
  } \\

  \center{\includegraphics[scale=.22]{ciencias.pdf}}
\end{minipage}
\begin{minipage}{.7\textwidth}
\flushright

\center{

  \center{
    \LARGE{U}\large{NIVERSIDAD} \LARGE{N}\large{ACIONAL}
    \LARGE{A}\large{UTÓNOMA} \\[10pt]
    \large{DE}
    \LARGE{M}\large{ÉXICO}
  } \\
  \rule{\textwidth}{2pt}
  \\
  \hrulefill\\[1cm]

  \LARGE{F}\large{ACULTAD DE } \LARGE{C}\large{IENCIAS}\\[2cm]

  \large{
  Análisis del algoritmo \textit{Knuth-Morris-Pratt} con énfasis en la programación funcional
  }\\[1.6cm]

  \huge{
T \hspace{1cm} E \hspace{1cm} S \hspace{1cm} I \hspace{1cm} S  }\\[1cm]

  \large{QUE PARA OBTENER EL TÍTULO DE:}\\[1cm]

  \large{
Licenciado en Ciencias de la Computación  }\\[1cm]

  \large{PRESENTA:}\\[1cm]

  \large{
Ángel Iván Gladín García  }\\[1cm]

  \large{
TUTORA  }\\[.2cm]

  \large{
  Dra. Lourdes Del Carmen Gonzales Huesca}\\[1cm]
  \large{
    Ciudad Universitaria, Cd. Mx., 2021
  }
}

\end{minipage}

    \clearpage
    \mbox{}
    \clearpage
    \thispagestyle{empty}
    
    \pagenumbering{Roman} 


\begin{center}
{\Large \textbf{Hoja de Datos del Jurado}}
\vspace*{.85cm}
\end{center}


\begin{enumerate}

\item Datos del Alumno

Gladín \\
García \\
Ángel Iván \\
+52 55 8196 8560 \\
Universidad Nacional Autónoma de México \\
Facultad de Ciencias \\
Ciencias de la Computación \\
313112470


\item Datos de la Tutora

Dra. \\
Lourdes del Carmen \\
González \\
Huesca


\item Datos del Sinodal 1

Dr. \\
Favio Ezequiel \\
Miranda \\
Perea


\item Datos del Sinodal 2

Dra. \\
Adriana \\
Ramírez \\
Vigueras


\item Datos del Sinodal 3

Dr. \\
Canek \\
Peláez \\
Valdés


\item Datos del Sinodal 4

L. en C.C. \\
Fernando Abigail \\
Galicia \\
Mendoza


\item Datos del trabajo escrito

{\small Análisis del algoritmo \textit{Knuth-Morris-Pratt} con énfasis en la programación funcional} \\
100p. \\ % TODO: cambiar el número de hojas
2021 


\end{enumerate} 

    
    \chapter*{Agradecimientos}
    \input{capitulos/agradecimientos.tex}
    \clearpage
    
    \tableofcontents


\mainmatter
    \chapter*{Motivación y estructura del trabajo}
        En mi camino aprendiendo y escribiendo programas usando programación funcional, es algo común ver un programa
que aunque sea corto y legible, muchas veces es algo ineficiente. Entonces es ahí cuando Richard Bird tiene
en mente que un programa debe actuar como la especificación formal del problema, pero también por medio del
razonamiento ecuacional poder calcular uno más eficiente.
Uno de los factores que ayudó al crecimiento en el interés de la programación funcional, fue que en los años
1990's se dieron cuenta que estos lenguajes son buenos para hacer razonamiento ecuacional.
De hecho el lenguaje funcional Gofer, inventado por Mark Jones capturó este pensamiento como un acrónimo 
(\textit{Good for equational reasoning}).
\newline

Lo que se abordará en este trabajo es primero empezar con una especificación en Haskell y después proseguir a
calcular una versión más eficiente por medio de razonamiento ecuacional.
La razón de este trabajo es ver hasta donde el diseño de un algoritmo puede estar encajado en una forma
matemática de calcular un resultando usando principios matemáticos bien establedicos como definiciones, 
teoremas, y \textit{``leyes''}.
Curiosamente, es generalmente verdadero que en matemáticas los cálculos están diseñados para simplificar
cosas complicadas, en el diseño de algoritmos usualmente es al revés.
\begin{quote}
Simples, pero ineficientes programas son transformados en versiones más eficientes que puedes ser
completamente opacas en su implementación.
\end{quote}
Explicando las ideas detrás de un algoritmo es mucho más fácil en un estilo funcional, en vez de un
procedimental. Las funciones pueden ser separadas fácilemente, cada una es sucinta y capturan patrones
de cómputo.
\newline

Los algoritmos de búsqueda de subcadenas (\textit{String Matching Algorithms}) son usados frecuentemente en:
programas de edición de texto para encontrar todas las ocurrencias de un patrón en un texto, para encontrar
patrones particulares en una cadenas de ADN, o también en algunos motores de búsqueda los utilizan para
encontrar páginas web en búsquedas, entre otras aplicaciones. Algoritmos efecientes para atacar este tipo de
problemas nos ayudan gratamente para mejorar el tiempo de la búsqueda.
\newline

TODO

Como lo menciona Richard S. Bird en su artículo \emph{Polymorphic String Matching}\cite{book:1505279}
% TODO: ponerle un formato bonito
El desarrollo de cálculos en programas funcionales ha sido asociado a trucos de magia: agradables de ver pero seguido pero a menudo hay un misterio en cómo se hacen.

En este trabajo se explicará esto, es dar un cálculo del algoritmo KMP,
%FIXME: quitar lo de abajo?
Este probleme de string matching está formulado polimórficamente, así que la única propiedad disponible que tienen los elementos del alfabeto es que sean comparables.

% TODO: de aquí me puedo sacar algunas ideas
% https://www.cs.princeton.edu/~rs/AlgsDS07/21PatternMatching.pdf

% TODO: poner que así organizo los .bib
% https://flamingtempura.github.io/bibtex-tidy/
    \addcontentsline{toc}{chapter}{Motivación y estructura del trabajo}
    
    \chapter{Fundamentos}
        En este capítulo se verá como, dado un conjunto de definiciones de funciones, por medio de razonamiento ecuacional podemos llegar a otras definiciones y/o probarlas. Las pruebas aquí se harán mediantte inducción.

Muchas veces es algo engorroso probar funciones similares repetidamente, por eso veremos una forma de hacer pruebas (en algunos casos) más cortas,
presentando unas \textit{funciones de orden superior} que encapsulan patrones comunes de cómputo. Y así, probar resultados más generales y apelar a ellos.

Al final se verá que la eficiencia también importa, porque se mostrarán algunos ejemplos; como un problema famoso llamado \textit{``The maximum segment sum''}
y una mejora de la función \texttt{scanr}. Y todo esto se logrará como consecuencia de lo dicho anteriormente.

\subsection{Inducción sobre listas}
Recordemos que toda lista finita es de la forma; una lista vacía \texttt{[]} ó \texttt{x:xs} donde \texttt{xs} es una lista finita. Por consiguiente, para probar que $P(xs)$ se mantiene para todas las lista finitas $xs$,
se tiene que probar que:

\begin{enumerate}
    \item $P([])$ se cumple
    \item Para toda $x$ y para todas las listas finitas $xs$, que $P(x:xs)$ se cumple dado que $P(xs)$ también.
\end{enumerate}

Tomemos la definición de concatenación \texttt{(++)},
\inputminted{haskell}{definiciones/concatenation.hs}

Y ahora probemos que (++) es asociativa para todas las listas finitas $xs$, es decir:

(xs ++ ys) ++ zs = xs ++ (ys ++ zs)

Por inducción sobre $xs$:
% TODO: poner aquí lo del strictness property https://stackoverflow.com/questions/27672585/efficient-version-of-inits/27674051#27674051

%{\displaystyle x+5} is the left-hand side (LHS) and {\displaystyle y+8}{\displaystyle y+8} is the right-hand side (RHS).
%FIXME: si caben dos eucaciones por lado lo hago en columnas.

\begin{itemize}
\item Caso []
\begin{minted}{haskell}
(LHS)

([] ++ ys) ++ zs
=   {++.1}
ys ++ zs
\end{minted}

\begin{minted}{haskell}
(RHS)

[] ++ (ys ++ zs)
=   {++.1}
ys ++ zs
\end{minted}

\item Caso (x:xs)
\begin{minted}{haskell}
(LHS)

((x:xs) ++ ys) ++ zs
=   {++.2}
(x:(xs ++ ys)) ++ zs
=   {++.2}
x:((xs ++ ys) ++ zs)
\end{minted}

\begin{minted}{haskell}
(RHS)

(x:xs) ++ (ys ++ zs)
=   {++.2}
x:(xs ++ (ys ++ zs))
=   {induction}
x:((xs ++ ys) ++ zs)
\end{minted}

\end{itemize}

\section{Pliegues}

\subsection{\texttt{foldr}}
\inputminted{haskell}{definiciones/foldr.hs}

\subsection{Síntesis de programas vía la propiedad universal}


\subsection{\texttt{foldl}}
\inputminted{haskell}{definiciones/foldl.hs}



\section{Programación funcional}
De forma muy general y resumiendo, la programación funcional:
\begin{itemize}
    \item es un método de construcción de un programa que hace énfasis en las funciones y sus aplicaciones
    en vez de cómandos y sus ejecuciones.

    \item usa notación matemática simple que permite que los problemas sean descritos de manera clara
    y consisa.
    \item tiene bases matemáticas que fundamentan el razonamiento ecuacional acerca de las propiedades de
    los programas.
\end{itemize}

\section{Definiciones inductivas y recursivas}
%TODO

\section{Razonamiento ecuacional}
%TODO


\section{Definiciones de listas}
%TODO

\section{Definiciones de funciones}
\inputminted{haskell}{definiciones/map.hs}



\section{Ley de Fusión}

\begin{minted}{haskell}
f . foldr g a = foldr h b
\end{minted}

\begin{itemize}
    \item $f$ es una función estricta.
    \item $f a = b$
    \item $f (g y x) = h (f y) x$ para toda $x$ y $y$.w
\end{itemize}

Ejmeplos:
\begin{itemize}
    \item \hsCode{double . sum    = foldr ((+) . double) 0}
    \item \hsCode{length . concat = foldr ((+) . length) 0}
\end{itemize}

% TODO poner lo de una función estrictca, está en el thinking functionally with hskell página 29

%% TODO: poner aquí lo de lados de la ecuación
% https://en.wikipedia.org/wiki/Sides_of_an_equation

% Polymorphic algorithms
% https://wiki.haskell.org/Polymorphism#:~:text=A%20value%20is%20polymorphic%20if,polymorphism%20and%20ad%2Dhoc%20polymorphism

% foldr
% https://wiki.haskell.org/Foldr_Foldl_Foldl'
%http://www.cantab.net/users/antoni.diller/haskell/units/unit06.html

\section{\textit{Scan Lemma}}

En esta sección se considerará la función \hsCode{scanl}, \hsCode{scanl} aplica un pliegue
izquierdo (\hsCode{foldl f e}) a cada segmento (todos los prefijos) de una lista como se muestra
a continuación,

\begin{minted}{haskell}
scanl (@) e [x, y, z, ...] = [e, e@x,(e@x)@y,((e@x)@y)@z,...]
\end{minted}

Se propondrá la siguiente especificación de \hsCode{scanl} como:
\begin{minted}{haskell}
scanl :: (b -> a -> b) -> b -> [a] -> [b]
scanl f e = map (foldl f e) . inits

inits :: [a] -> [[a]]
inits []     = [[]]
inits (x:xs) = [] : map (x:) (inits xs)
\end{minted}

Ejemplo:
\begin{minted}{haskell}
>>> scanl (+) 0 [1..10]
[0,1,3,6,10,15,21,28,36,45,55]
\end{minted}

La expresión anterior calcula la suma de cada prefijo de una lista de números del 1 al 10.
\begin{minted}{haskell}
[0, 0+1, (0+1)+2, ((0+1)+2)+3, (((0+1)+2)+3)+4, ...]
\end{minted}

Veamos un ejemplo de \hsCode{scanl}
\begin{minted}{haskell}
>>> inits [1..5]
[[],[1],[1,2],[1,2,3],[1,2,3,4],[1,2,3,4,5]]
\end{minted}

Pero se puede ver que la definición propuesta de \hsCode{scanl} involucra evaluar \hsCode{f} un
total de $\sum_{i=0}^{n} i = \frac{n(n+1)}{2}$ veces sobre una lista de longitud $n$.

Es aquí cuando uno se pregunta si ¿se podría hacer mejor?. La respuesta es sí, cálculando una mejor
definiciónpor medio de razonamiento ecuacional. Cosideremos por casos.

\begin{itemize}
\item Caso \hsCode{[]}
\begin{minted}{haskell}
scanl f e []
  = -- {Definición de scanl}
map (foldl f e) (inits [])
  = -- {Por inits.1}
map (foldl f e) [[]]
  = -- {Por map.1 y map.2}
(foldl f e []) : map (foldl f e) []
  = -- {Por foldl.1 y map.1}
e : []
  = -- {Azucar sintáctica}
[e]
\end{minted}

Teniendo así que \hsCode{scanl f e [] = [e]}.

\item Caso \hsCode{x:xs}
\begin{minted}{haskell}
scanl f e (x:xs)
  = -- {Definición de scanl}
map (foldl f e) (inits (x:xs))
  = -- {Por inits.2}
map (foldl f e) ([] : map (x:) (inits xs))
  = -- {Por map.1 y map.2}
(foldl f e []) : (map (foldl f e . (x:)) (inits xs))
  = -- {Por foldl.1}
e : (map (foldl f e . (x:)) (inits xs))
  = -- {Por la afrimación que se demostrará abajo y que es el caso (x:xs)}
e : map (foldl f (f e x)) (inits xs)
  = -- {Por la primera definición de scanl.1}
e:scanl f (f e x)
\end{minted}

\textbf{Afirmación:} \hsCode{foldl f e . (x:) = foldl f (f e x)}. Se seguirá como una consecuencia
inmediata de \hsCode{foldl}.

\begin{itemize}
\item Caso \hsCode{x:[]}
\begin{minted}{haskell}
foldl f e . (x:) []
  = -- {Composición de funciones}
foldl f e [x]
  = -- {Aplicación y azucar sintáctica}
foldl f (f e x) []
  = -- {Por foldl.1 y Por foldl.2}
e
\end{minted}

\item Caso \hsCode{x:ys}, análogo al anterior.
\end{itemize}
\end{itemize}

Teniendo así una nueva definición de \hsCode{scanl} como,
\begin{minted}{haskell}
scanl f e []     = [e]
scanl f e (x:xs) = e:scanl f (f e x) xs
\end{minted}

donde \hsCode{f} solo se calcula un número lineal de veces, a diferencia de su primera definición
propuesta que requería un número cuadrático de veces.

Aunque si vemos definición\footnote{
\url{https://hackage.haskell.org/package/base-4.14.1.0/docs/src/GHC.List.html\#scanl}
} del preludio es diferente:
\begin{minted}{haskell}
scanl                   :: (b -> a -> b) -> b -> [a] -> [b]
scanl                   = scanlGo
  where
    scanlGo           :: (b -> a -> b) -> b -> [a] -> [b]
    scanlGo f q ls    = q : (case ls of
                               []   -> []
                               x:xs -> scanlGo f (f q x) xs)
\end{minted}

Pero esto de debe a que la versión que se calculó da \hsCode{scanl f e undefined = undefined} y la
versión de preludio \hsCode{scanl f e undefined = e:undefined}. Esto se debe a que como Haskell
es perezoso, no nos debemos de preguntar nada acerca de la lista a procesar, pero lo que es seguro
es que empieza con \texttt{e}.

En general, cualquier problema que involucre la función \hsCode{inits}, este lema es bastante útil
de saber porque si recordamos la primera especificación:
\begin{minted}{haskell}
scanl f e = map (foldl f e) . inits
\end{minted}

La LHS toma $\Theta(n)$ el número de evaluaciones de \hsCode{f} mientras que RHS toma $\Theta(n^2)$.
Y como se demostró ambas expresiones son equivalentes.

\begin{figure}[h]
\caption{Un ejemplo concreto de la versión propuesta inicialmente de \texttt{scanl} y la que se derivó}
\centering
\includegraphics[width=0.9\textwidth]{scan_lemma_example.pdf}
\end{figure}

\subsection{\textit{The maximum segment sum}}

\section{\textit{Tupling}}
\section{\textit{Strict property}}


% TODO: aquí agarrar cosas del polymorphic en la parte que dice: on compositioin https://wiki.haskell.org/Tutorials/Programming_Haskell/String_IO
    
    \chapter{Análisis de tiempo}
        \lipsum[1-1]

\section{Notación asintótica}
%TODO

\section{Estimando tiempo}
%TODO

\section{Tiempo amortizado}
%TODO

    
    \chapter{Algoritmo de búsqueda de subcadenas}
        \lipsum[1-1]

\section{Motivación}
\lipsum[2-4]

\section{Notación y terminología}
\lipsum[2-4]

\section{Algoritmo de búsqueda de subcadenas ingenuo (\textit{naïve})}
\lipsum[2-4]

\section{Diferentes tipos de algoritmos en cadenas}
\lipsum[2-4]

    
    \chapter{Análisis del la versión funcional}
        % Quizá aquí poner una sección de introducción ???

Como ya hemos explicado anteriormente, el problema de \textit{string matching} se trata de buscar todas las ocurrencias de una cadena no vacía, que se le denominará \textbf{patrón} en un texto llamado \textbf{texto}.
Empecemos con una especificación muy básica, por ahora sin preocuparnos en la complejidad,

\begin{minted}{haskell}
matches :: Eq a => [a] -> [a] -> [Int]
matches ws = map length . filter (endswith ws) . inits
\end{minted}

Donde definiremos a \hsCode{endswith} como:

\begin{minted}{haskell}
endswith ws xs = ws `elem` tails xs
\end{minted}

La función \hsCode{inits} regresa una lista los prefijos de una lista en orden creciente como se puede ver a continuación
\begin{minted}{haskell}
> inits [2,3,5]
  [[],[2],[2,3],[2,3,5]]
\end{minted}

La función \hsCode{endswith ws xs} checa si el patrón \hsCode{ws} es un sufijo de \hsCode{xs}.
Entonces \hsCode{matches ws xs} regresa una lista de enteros donde cada entero \hsCode{n} indica que \hsCode{ws} aparece en \hsCode{xs} terminando en posición \hsCode{n} (siendo 1-indexados). Por ejemplo:

\begin{minted}{haskell}
> matches "abcab" "ababcabcab"
  [7, 10]
\end{minted}

La función \hsCode{matches} es polimórfica, así que cualquier algoritmo tiene que depender en esta función de igualdad \hsCode{(==) :: a -> a -> Bool} acerca de elementos de dos listas.
Asumiendo que toma tiempo constante hacer la función de igualdad, la complejidad en tiempo de \hsCode{matches ws xs} es de $\Theta(mn)$ pasos en el peor de los casos, donde $m$ el la longitud de \hsCode{ws} y $n$ la de \hsCode{xs}.

% TODO, explicar porqué esa complejidad y checar todo lo que escribí hasta aquí. 

% Quizá mover esto hasta arriba ???
Como se mencionó en la motivación de este trabajo, lo que se hará es dar una especificación en Haskell para
despues irlo mejorando. Empecemos con esta definición

% FIXME: está bien culero explicado, mejorar
Por ahora se debe de tener en mente que se debe buscar funciones \hsCode{p} y \hsCode{op} y un valor \hsCode{e},
tal que \hsCode{endswith ws = p . foldl op e}.

Entonces se tiene que,

\inputminted{haskell}{definiciones/kmp/1-matches.hs}

Donde \hsCode{matches ws xs} es una lista de enteros, en la cual cada entero \hsCode{n} dice si el patrón \hsCode{ws} aparece en el texto
\hsCode{xs} acabando en la posición $n$.

Dadas \hsCode{p} y \hsCode{op} que tomen tiempo constante, a al menos tiempo constante amortizado, \texttt{matches} tomará $\Theta(n + m)$ pasos, donde
el patrón tiene una longitud de $m$ y el texto una longitud de $n$.

%% TODO, hasta aquí todo luce bien. checarlo

\section{Primer acercamiento}

Se puede escribir a \hsCode{endswith ws} como una composición funciones,

\begin{minted}{haskell}
endswith ws = not . null . filter (= ws) . tails
\end{minted}

Pero no es una buena idea definir a \hsCode{filter (= ws) . tails} con una función de plegado \hsCode{foldl}
porque regresaria o bien una lista vacia o \hsCode{[ws]}, lo cual sería información insuficiente
para definir esa función inductivamente.

Para ejemplificar esto hagamos esa función,
\begin{minted}{haskell}
f ws = foldl (\ys xs -> if xs == ws then xs:ys else ys) [] . tails
\end{minted}

\begin{minted}{haskell}
> f "nea" "Atenea"
  ["nea"]

> f "ene" "Atenea"
  []
\end{minted}

Consideremos la función prefijo $(\sqsubseteq)$ definida como:

\inputminted{haskell}{definiciones/prefix.hs}

Teniendo la función prefijo, se ve más prometedora la función \hsCode{filter (⊑ ws) . tails} porque
se podría obtener más ``información'' inductivamente.

Donde \hsCode{tails} está definida como,

\inputminted{haskell}{definiciones/tails.hs}

Ejemplo,

\begin{minted}{haskell}
> tails [2,3,5,7]
  [[2,3,5,7],[3,5,7],[5,7],[7],[]]
\end{minted}

Aplicando \hsCode{xs} a la función \hsCode{filter (⊑ ws) . tails}, regresará en orden decreciente sobre la longitud de
todas la colas de \hsCode{xs} los que sean prefijos de \hsCode{ws}.

% Poner ejemplo

Entonces, el primer elemento de la lista es \hsCode{ws} si y solo si \hsCode{endswith ws xs} es verdadero. De ahí que,

\begin{minted}{haskell}
endswith ws = (= ws) . head . filter (⊑ ws) . tails
\end{minted}

%%%%%%% TODO: HASTA AQUÍ ESTÁ MEDIO BIEN

La primera función \hsCode{(= ws)} no es función que tome tiempo constante (por obvias razones). Ese problema es resulto generalizando la función
\hsCode{filter (⊑ ws) . tails} a una función \hsCode{split} definida como,

\begin{minted}{haskell}
split ws xs = head [(us, ws ↓ us) | us <- tails xs, us ⊑ ws]
\end{minted}

La función \hsCode{↓} de forma abstracta está definida como \hsCode{(us + vs) ↓ us = vs}. De ahí que \hsCode{split ws xs} separa \hsCode{ws} (el patrón)
en dos listas \hsCode{us} y \hsCode{vs} tal que \hsCode{us ++ vs = ws}.

\inputminted{haskell}{definiciones/kmp/1-down-arrow.hs}

\begin{center}
\textbf{El valor de \hsCode{us} es el sufijo más largo de \hsCode{xs} que es prefijo de \hsCode{ws}.}
\end{center}

Por ejemplo,
\begin{minted}{haskell}
>>> split "endnote" "append" = ("end", "note")
\end{minted}

% TODO: seguir estándar de ejecución del libro
\begin{minted}{haskell}
>>> split "endnote" "append" = ("end", "note")
    head [(us, ws ↓ us) | us <- ["append","ppend","pend","end","nd","d",""], us ⊑ ws]
    head [("end","note"), ("","endnote")]
    ("end","note")
\end{minted}

% FIXME: Muy repetitivo, mejorar

Ahora tenemos que \hsCode{endswith ws = null . snd . split ws}.
Queda a encontrar \hsCode{op} y \hsCode{e} tal que \hsCode{split ws = foldl op e}.
De forma equivalente, se quiere encontrar \hsCode{e} y \hsCode{op} tal que satisfagan, 

\inputminted{haskell}{definiciones/kmp/1-split-eq.hs}

% Explicar por que split ws [] = ([], ws) = e
Se tiene que \hsCode{split ws [] = ([], ws)}, lo que nos da \hsCode{e}. Falta descubrir \hsCode{op}.

La siguiente observación es crucial,

\hsCode{split ws xs = (us,vs)} $\quad\Longrightarrow\quad$ \hsCode{split ws (xs ++ [x]) = split ws (us ++ [x])}

Que se interpreta como, el sufijo más largo de \hsCode{xs ++ [x]} es un prefijo de \hsCode{ws} es un sufijo de \hsCode{us ++ [x]}.

Para descubiri \hsCode{op}, primero se debe expresar e a \hsCode{split} recursivamente,

\begin{minted}{haskell}
split ws xs = if xs ⊑ ws then (xs, ws ↓ xs) else split ws (tail xs)
\end{minted}

% Explicar por que es equivalente a la versión por comprensión de listas

Ahora, como se sabe que \hsCode{split ws xs = (us, vs)} y que \hsCode{ws = us ++ vs}, se puede hacer el razonamiento:

\inputminted{haskell}{definiciones/kmp/1-op-raz-eq.hs}

Y todo ese cálculo nos da la definición de \hsCode{op :: ([a], [a]) -> a -> ([a], [a])} como:

\inputminted{haskell}{definiciones/kmp/1-op.hs}

Quedando así (la primera versión) hasta ahora:

\inputminted{haskell}{definiciones/kmp/1-final.hs}

% TODO poner la explicación que falta justo antes de Data refinement

%% FIXME: cambiar esto por algo más aca chido
evaluación  la función \hsCode{matches} se ve así:

% Poner el ejmeplo y su ejecución

Juntando todo lo anterior, el primer acercamiento que se puede utilizar quedaría como:

\inputminted{haskell}{codigo/haskell/1-first-steps.hs}

\section{Refinamiento de los datos}

\begin{minted}{haskell}
abs :: Rep ([a], [a]) -> ([a], [a])
rep :: ([a], [a]) -> Rep ([a], [a])
\end{minted}


\begin{minted}{haskell}
foldl op ([], ws) = abs . foldl op' (rep ([], ws))  -- (17.1)
\end{minted}

\begin{minted}{haskell}
matches ws = map fst . filter (null . snd . abs . snd) .
             scanl step (0, rep ([], ws))
step (n, r) x = (n + 1, op' r x)    
\end{minted}

\begin{minted}{haskell}
op' r = rep . op (abs r)
\end{minted}

\begin{minted}{haskell}
abs (op' r x) = abs (rep (op (abs r) x)) = op (abs r) x
\end{minted}

\begin{minted}{haskell}
op' r x | [x] ⊑ vs = rep (us ++ [x],tail vs)
        | null us = rep ([],ws)
        | otherwise = op' (rep (split ws (tail us))) x
        where (us, vs) = abs r
\end{minted}

\section{Árboles}

\begin{minted}{haskell}
data Rep a = Null | Node a (Rep a) (Rep a)
\end{minted}

\begin{minted}{haskell}
abs (Node (us, vs) lr) = (us,vs) -- (17.3)
\end{minted}

\begin{minted}{haskell}
rep (us, vs) = Node (us, vs) (left us vs) (right us vs) -- (17.4)
\end{minted}

\begin{minted}{haskell}
left [] vs = Null
left (u : us) vs = rep (split ws us)
right us [] = Null
right us (v:vs) = rep(us ++ [v], vs)
\end{minted}

\begin{minted}{haskell}
op' (Node (us, vs) l r) x | [x] ⊑ vs = r
                          | null us = root
                          | otherwise = op' l x
\end{minted}

\begin{minted}{haskell}
op' (Node (us, vs) l r) x
= {definition of op' in the case [x] ⊑ vs}
rep (us ++ [x], tail vs)
= {definition of right and x = head vs}
right us vs
= {definition of rep}
r
\end{minted}

\begin{minted}{haskell}
op' Null x = root
op' (Node (us, vs) l r) x | [x] ⊑ vs = r
                          | otherwise = op' l x
\end{minted}

\begin{minted}{haskell}
rep (us, vs) = grep (left us vs) (us, vs)
\end{minted}

\begin{minted}{haskell}
right us (v:vs) = rep (us ++ [v], vs)
                = grep (left (us ++ [v]) vs) (us ++ [v], vs)
\end{minted}

\begin{minted}{haskell}
left ([] ++ [v]) vs
= {definition of left}
rep (split ws [])
= {definition of split}
rep ([], ws)
= {definition of root}
root
\end{minted}

\begin{minted}{haskell}
left (u : us ++ [v]) vs
= {definition of left}
rep (split ws (us ++ [v]))
= {definition of split}
rep (op (split ws us) v)
= {definition (17.2) of op'}
op' (rep (split ws us)) v
= {definition of left}
op' (left (u : us) vs) v
\end{minted}

\begin{minted}{haskell}
left (us ++ [v]) vs = if null us then root else op' (left us vs) v
\end{minted}

\begin{minted}{haskell}
grep l (us, []) = Node(us, []) l Null
grep l (us, v:vs) = Node(us, v:vs) l
                    (grep (op' l v) (us ++ [v], vs))
\end{minted}

%% HASTA AQUÍ CODIGO PAG 133!!!

%\section{Ejemplos}

    
    \chapter{Algoritmo Knuth-Morris-Pratt}
        \lipsum[1-1]

\section{Versión imperativa}
\inputminted{cpp}{codigo/cpp/kmp.cpp}


\section{Versión funcional}
\inputminted{haskell}{codigo/haskell/3-trees.hs}

    
    \chapter{Benchmarks}
        \input{capitulos/6-benchmarks}

    \chapter{Conclusiones}
        \input{capitulos/7-conclusiones}
    


\backmatter
    \nocite{*}
    \printbibliography

\end{document}
